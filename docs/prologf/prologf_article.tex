%%%%%%%%%%%%%%%%%%%%%%%%%%%%%%%%%%%%%%%%%
% Journal Article
% LaTeX Template
% Version 1.4 (15/5/16)
%
% This template has been downloaded from:
% http://www.LaTeXTemplates.com
%
% Original author:
% Frits Wenneker (http://www.howtotex.com) with extensive modifications by
% Vel (vel@LaTeXTemplates.com)
%
% License:
% CC BY-NC-SA 3.0 (http://creativecommons.org/licenses/by-nc-sa/3.0/)
%
%%%%%%%%%%%%%%%%%%%%%%%%%%%%%%%%%%%%%%%%%

%----------------------------------------------------------------------------------------
%	PACKAGES AND OTHER DOCUMENT CONFIGURATIONS
%----------------------------------------------------------------------------------------

\documentclass[twoside,twocolumn]{article}

\usepackage{blindtext}                                          % Package to generate dummy text throughout this template

\usepackage[sc]{mathpazo}                                       % Use the Palatino font
\usepackage[T1]{fontenc}                                        % Use 8-bit encoding that has 256 glyphs
\linespread{1.05}                                               % Line spacing - Palatino needs more space between lines
\usepackage{microtype}                                          % Slightly tweak font spacing for aesthetics

\usepackage[english]{babel}                                     % Language hyphenation and typographical rules

\usepackage[hmarginratio=1:1,top=32mm,columnsep=20pt]{geometry} % Document margins
\usepackage[hang, small,labelfont=bf,up,textfont=it,up]{caption} % Custom captions under/above floats in tables or figures
\usepackage{booktabs}                                           % Horizontal rules in tables

\usepackage{lettrine}                                     % The lettrine is the first enlarged letter at the beginning of the text

\usepackage{enumitem}                                           % Customized lists
\setlist[itemize]{noitemsep}                                    % Make itemize lists more compact

\usepackage{abstract} % Allows abstract customization
\renewcommand{\abstractnamefont}{\normalfont\bfseries}          % Set the "Abstract" text to bold
\renewcommand{\abstracttextfont}{\normalfont\small\itshape}     % Set the abstract itself to small italic text

\usepackage{titlesec}                                           % Allows customization of titles
\renewcommand\thesection{\Roman{section}}                       % Roman numerals for the sections
\renewcommand\thesubsection{\roman{subsection}}                 % roman numerals for subsections
\titleformat{\section}[block]{\large\scshape\centering}{\thesection.}{1em}{} % Change the look of the section titles
\titleformat{\subsection}[block]{\large}{\thesubsection.}{1em}{} % Change the look of the section titles

\usepackage{fancyhdr}                                           % Headers and footers
\pagestyle{fancy}                                               % All pages have headers and footers
\fancyhead{}                                                    % Blank out the default header
\fancyfoot{}                                                    % Blank out the default footer
\fancyhead[C]{PrologPF: Adding functions to Prolog}             % Custom header text
\fancyfoot[RO,LE]{\thepage}                                     % Custom footer text

\usepackage{titling}                                            % Customizing the title section

\usepackage{hyperref}                                           % For hyperlinks in the PDF

\usepackage{listings}                                           % For code blocks
\lstset{language=Prolog}

%----------------------------------------------------------------------------------------
%	TITLE SECTION
%----------------------------------------------------------------------------------------

\setlength{\droptitle}{-4\baselineskip}                     % Move the title up

\pretitle{\begin{center}\Huge\bfseries}                     % Article title formatting
\posttitle{\end{center}}                                    % Article title closing formatting
\title{PrologPF: Adding functions to Prolog}                % Article title
\author{%
\textsc{Ian Lewis}\thanks{A thank you or further information} \\[1ex] % Your name
\normalsize University of Cambridge \\                      % Your institution
\normalsize \href{mailto:ijl20@cam.ac.uk}{ijl20@cam.ac.uk}  % Your email address
%\and                                                       % Uncomment if 2 authors are required, duplicate these 4 lines if more
%\textsc{Jane Smith}\thanks{Corresponding author} \\[1ex]   % Second author's name
%\normalsize University of Utah \\                          % Second author's institution
%\normalsize \href{mailto:jane@smith.com}{jane@smith.com}   % Second author's email address
}
\date{\today}                                               % Leave empty to omit a date
\renewcommand{\maketitlehookd}{%
\begin{abstract}
\noindent
PrologPF is a language that combines the logic programming of Prolog with the
functional reduction of the language ML. The syntax has been pragmatically designed to
provide a consistent combined style that would be familiar to both Prolog and ML programmers.
Declared functions can be called naturally from within Prolog clauses, i.e. a function reference
can appear as the functor of a compound term resulting in functional reduction of that term.
Prolog relations can be called with one-solution semantics as the condition in functional
if-then-else statements. Unification is used consistently for argument matching in both relations
and functions, and unifications from the functional if-then-else condition are propagated into the
then and else statements. Higher-order functional programming and lazy reduction of
functions are supported. The implementation is designed as an extension of Prolog, and has been
shown to support all of the ML programming exercises in an undergraduate CS course. The PrologPF approach
to blending functional and logic programming combines the declarative nature of both paradigms while
replacing deterministic relations with declarative function definitions and reducing the need
for Prolog's extra-logical \textit{cut}.
\end{abstract}
}

%----------------------------------------------------------------------------------------

\begin{document}

% Print the title
\maketitle

%----------------------------------------------------------------------------------------
%	ARTICLE CONTENTS
%----------------------------------------------------------------------------------------

\section{Introduction}

Prolog code tends to be a mix of relations and functions although the limited
syntax requires the functions to be expressed using the relational head-normal form.

For example, a factorial function in Prolog will typically be written

\begin{verbatim}
fact(0,1).
fact(N,F) :-
    N > 0,
    M is N - 1,
    fact(M,FM),
    F is N * FM.
\end{verbatim}

Note this relation is deterministic and requires the first argument to be instantiated
to a number without those details being clearly expressed. More problematically, the
implementation of functions as relations encourages the use of \textit{cut}, as in

\begin{verbatim}
fact(0,1) :- !.
fact(N,F) :-
    M is N - 1,
    fact(M,FM),
    F is N * FM.
\end{verbatim}

The extra-logical behaviour of \textit{cut} destroys the declarative nature of Prolog
code and in this case silently introduces non-terminating behaviour when the first
argument is a negative number.

\section{Method}

In PrologPF, the factorial function can be defined as below.
                                                    % Start code-block
\begin{verbatim}
fun fact(1) = 1;
    fact(N) = N * fact(N-1).
\end{verbatim}

This style of the function definition has familiarity to both Prolog's alternate head terms
for typical relation declarations and the alternate argument matching cases in Standard ML.

As PrologPF has special treatment for if-then-else in function definitions, the
factorial function can equally be written using this style.
                                                    % Start code-block
\begin{verbatim}
fun fact(N) = if (N=1)
              then 1
              else N * fact(N-1).
\end{verbatim}

Function references can occur naturally in place of Prolog terms, for example
\begin{verbatim}
% fact_list(+L,?FL) succeeds if FL is
% a list of factorials from numbers L.
fact_list([],[]).
fact_list([H|T],[fact(H)|FT])
    :- fact_list(T,FT).
\end{verbatim}

In PrologPF, the operator @ is used to express functional application, i.e.
\begin{verbatim}
    fact(5)
\end{verbatim}
where fact has been declared as a function using the fun operator, is equivalent to
\begin{verbatim}
    fact @ [5]
\end{verbatim}

%------------------------------------------------

\section{Methods}

Maecenas sed ultricies felis. Sed imperdiet dictum arcu a egestas.
\begin{itemize}
\item Donec dolor arcu, rutrum id molestie in, viverra sed diam
\item Curabitur feugiat
\item turpis sed auctor facilisis
\item arcu eros accumsan lorem, at posuere mi diam sit amet tortor
\item Fusce fermentum, mi sit amet euismod rutrum
\item sem lorem molestie diam, iaculis aliquet sapien tortor non nisi
\item Pellentesque bibendum pretium aliquet
\end{itemize}
\blindtext % Dummy text

Text requiring further explanation\footnote{Example footnote}.

%------------------------------------------------

\section{Results}

\begin{table}
\caption{Example table}
\centering
\begin{tabular}{llr}
\toprule
\multicolumn{2}{c}{Name} \\
\cmidrule(r){1-2}
First name & Last Name & Grade \\
\midrule
John & Doe & $7.5$ \\
Richard & Miles & $2$ \\
\bottomrule
\end{tabular}
\end{table}

\blindtext % Dummy text

\begin{equation}
\label{eq:emc}
e = mc^2
\end{equation}

\blindtext % Dummy text

%------------------------------------------------

\section{Discussion}

\subsection{Subsection One}

A statement requiring citation \cite{Figueredo:2009dg}.
\blindtext % Dummy text

\subsection{Subsection Two}

\blindtext % Dummy text

%----------------------------------------------------------------------------------------
%	REFERENCE LIST
%----------------------------------------------------------------------------------------

\begin{thebibliography}{99} % Bibliography - this is intentionally simple in this template

\bibitem[Figueredo and Wolf, 2009]{Figueredo:2009dg}
Figueredo, A.~J. and Wolf, P. S.~A. (2009).
\newblock Assortative pairing and life history strategy - a cross-cultural
  study.
\newblock {\em Human Nature}, 20:317--330.

\end{thebibliography}

%----------------------------------------------------------------------------------------

\end{document}
