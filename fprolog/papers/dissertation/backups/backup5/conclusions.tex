\chapter{Conclusions}
\label{conclusions}

Programs compiled with the PrologPF compiler can exploit the processing
power available in a LAN or WAN of general purpose workstations, without
requiring programmer annotations to the code to guide the parallelisation.

The breadth-first partitioning strategy minimises the communication
necessary for distributed execution, such that the technique is
optimised for systems with many processors but great communications
delay, such as the Internet.  The use of oracles provides an efficient
means of recovery from path processor failure, supporting the use of
the technique in a widely distributed system.  The single systems
image provided by the combination of the PrologPF compiler and the
skynet control system transform a generally idle network of
workstations into a usable super-computer.

PrologPF provides effective speedup for large\footnote{In this context, \textit{large}
implies a single-cpu runtime greater than a few seconds.}
problems containing sufficient or-parallelism.  For smaller problems, the runtime
taken to generate sufficent oracles for distribution to the available path processors
will dominate the overall runtime.

Given a large problem, the speedup achieved by PrologPF is determined by the
evenness of the balance of the workload assigned to the path processors.  The optimal
depth limit at which partitioning will take place will generate a set of oracles large
enough to assign many open oracles to each path processor.

The implementation of the oracle support in PrologPF imposes a 9 to 16 per cent
overhead on the execution of the compiled program on a single cpu.  This overhead
may be acceptable for the use of the compiled binaries in both a single-cpu and
distributed environment.

The initial phase of the breadth-first partitioning strategy produces
a number of open oracles at the selected depth.  The distribution of
work beneath these oracles can be quantified with the simulation of 
a subsequent execution of the problem with the number of path processors
$G$ set to the number of open oracles $S$ discovered at the depth limit
$L$ in an earlier execution with $G=1$.
An analysis of some commonly available Prolog benchmark programs shows
that many of those oracles will refer to small subtrees, with
a sparse distribution of oracles referring to large subtrees
representing most of the available work.
To achieve effective partitioning of the search tree, the depth limit
at which open oracles are found must generate a sufficient number of
oracles referring to large subtrees, such that all path processors can
expect to receive a number of these sizeable oracles.  The general
issue for oracle-based scheduling strategies such as the 
breadth-first partitioning used in PrologPF is that of the
\textit{poisoned oracle}.  For any strategy without subsequent work
splitting, a poisoned oracle is likely to be an oracle which refers
to a very large subtree, such that the uninterrupted sequential search of
that subtree dominates the overall parallel runtime.  For other
strategies used with oracles requiring communication, 
such as the automatic partitioning strategy of DelphiKS, a poisoned oracle
is that referring to a very small or nil subtree.

An effective work estimation function could lead to better balancing of
the work assignment to the path processors.  The generally sparse
and random distribution of the large oracles means that work estimation
based upon the arithmetic mean of the oracle neighbours will not
succeed.  Partially searching the subtree but limiting the number of
choicepoints traversed similarly relies upon a limited measure of
workload beneath certain oracles being used as an estimate for others.

The breadth-first partitioning strategy of PrologPF succeeds because
a large number of oracles can be generated and allocated, and the
process of following an assigned oracle within a path processor is
very efficient.

The metalogical predicate \textit{cut}\footnote{Written \texttt{!} in Prolog.}
 used in sequential Prolog programs to
prune the search tree does not have an efficient implementation in a
distributed or-parallel environment.  The propagation of a \textit{cut} in the search
tree of one path processor requires the communication to the other path
processors searching affected subtrees.  A program containing a small procedure
with \textit{cut} may encounter that predicate thousands of times each second.
PrologPF provides a useful programming system for logic programs that in Prolog
would contain \textit{cut} by supporting deterministic execution through the
use of higher-order functional programming, explicit negation through the use
of boolean functions, and permitting \textit{cut} in deterministic relational
procedures.

The implementation of higher-order functions in PrologPF has been achieved while
maintaining upwards compatibility with standard Prolog.  A consistent syntax is
possible for the combined styles, permitting reduction of functional terms as
arguments to relations and requiring relational goals as conditions in functions.
Allowing the definition of functions and relations of the same name but different arities
within the same PrologPF program facilitates the straightforward mapping of existing
deterministic library relations into new functions.  All function calls in PrologPF
appear in arguments to relations. Function failure can be supported as an exception
propagated to the outermost call in the argument term and
treated as unification failure leading to the failure of the underlying relational
subgoal.

PrologPF produces efficient compiled code, with the compiler built upon the
\texttt{wamcc} Prolog to C compiler \todo{cite wamcc}.  Efficient support for
oracles can be provided though the use of two primitives, \texttt{o\_{}build}
and \texttt{o\_{}follow}, embedded in the user program.  The definition of the
primitives permits their inline implementation as C macros.

For the specific, simple, breadth-first partitioning scheduling algorithm used by
PrologPF, an equivalent behaviour can be achieved through the use of a novel primitive
\texttt{kappa}.  The primitive, embedded as a proposition in the user program, performs
dynamic pruning resulting in the same partitioning of the search as PrologPF but without
the intermediate use of oracles.  The absence of oracles means that basic refinements
to the technique available to a future development of PrologPF may not be feasible with
\texttt{kappa}.  Implemented as a two-line C macro, \texttt{kappa} can be used with
most standard Prolog compilers.

%%%%%%%%%%%%%%%%%%%%%%%
\section{Future work} %
%%%%%%%%%%%%%%%%%%%%%%%

PrologPF requires the user to select a suitable depth limit for the partitioning
of a given problem.  Analysis is needed to determine whether an iterative first
phase might arrive at a reasonable value for the depth limit automatically, or
whether compile-time analysis is essential.

The scheduling strategy used in PrologPF assumes the work beneath the
discovered open oracles is randomly distributed, and no attempt is made to
associate a work estimate with each oracle.  If an effective estimation technique
could be found, then the effectiveness of the one-time allocation of work would be
greatly improved.

With the existing imbalance in the workloads assigned to the path processors, more
effective exploitation of the available processing power could be achieved if
the workload of busy processors were split between those that have become idle.
If recomputation is to be minimised, this requires interruption of the busy
path processor and communication of its current oracle back to a control processor.
Suitable strategies must be devised to determine the interrupt criteria,
and the re-partitioning technique.  As work splitting reduces the impact of poor
initial partitioning, the requirements of the first phase may differ from those in
PrologPF.

In PrologPF, the parallelisation of the user program is limited to the 
logical component.  Functional reduction within a specific subgoal is executed
sequentially.  Further research may lead to an effective method for extending the
use of oracles to parallelise functional execution.

PrologPF supports the use of \textit{failure} as an uncatchable exception within
a functional reduction, which is propagated to the top level function call and
leads to failure of the underlying subgoal.  More flexible support might be
provided through the general provision of exception support in the functions,
unified with the exception support of standard Prolog.






