PrologPF is a parallelising compiler targetting a distributed system of general purpose
workstations connected by a relatively low performance network.  The source language
extends standard Prolog with the integration of higher-order functions.

The execution of a compiled PrologPF program proceeds in a similar manner to
standard Prolog, but uses \textit{oracles} in one of two modes.  An oracle
represents the sequence of clauses used to reach a given point in the problem
search tree, and the same PrologPF binary can be used to \textit{build} oracles,
or \textit{follow} oracles previously generated.

The parallelisation strategy used by PrologPF proceeds in two phases.  An
initial phase searches the problem tree to a limited depth, recording the oracles
representing the discovered incomplete paths.  
In the second phase these oracles are allocated, with
the complete program, to the available processors in the network.  Each processor
follows its assigned oracles and fully searches the subtree referenced by each
oracle, sending solutions back to a control processor.  Processor failure can
be accomodated by the reassignment of the affected oracles.

For a problem requiring \textit{all} solutions to be found, execution completes
when all the distributed processors have completed the search of their assigned
subtrees.  If \textit{one} solution is required, the execution of all the path
processors is terminated when the control processor receives the first
solution.

The presence of the metalogical Prolog predicate \textit{cut} in the user
program conflicts with the use of oracles to represent valid open subtrees.
PrologPF promotes the use of higher-order functional programming as an
alternative to the use of \textit{cut}.  The combined language shows that
functional support can be added as a consistent extension to standard Prolog.
