\chapter{Higher-Order Functions in PrologPF}
\label{functions}

\section{Introduction}

The earlier implementations of parallel Prolog exploiting the Delphi
principle (\cite{CA87,Clo87,Wre90,Kle91,Sar95})
can support programs written in a pure subset of Prolog (see Chapter
\ref{cut}).

PrologPF extends the Delphi Machine to allow the use of \textit{cut},
but only for
deterministic procedures.  The programmer must avoid the
intentional or accidental use of \textit{cut} within
procedures which still (in spite of the \textit{cut}) have
multiple solutions.  

However, the need for \textit{cut} within a
PrologPF program is greatly reduced as
support is included for the definition and application
of functions, in which the deterministic
execution is ensured by the system.  Also, boolean functions can often
be used where Prolog would rely upon the use of failure to express negation.

The higher-order functional support in PrologPF is sufficient to allow
straightforward programming of all the exercises in an undergraduate ML
functional programming course \cite{Pau88}, and to allow a version of the SRI Prolog 
Technology Theorem Prover \cite{Sti88} to be implemented without \textit{cuts}
(Chapters \ref{case_fcs} and \ref{case_pttp}).

PrologPF extends Prolog with support of the definition and deterministic
evaluation of higher-order functions, with the functions treated as
first-class values within the logic system.  The Delphi oracles do not
extend into the functional reduction graph, and no parallelism is provided for
the evaluation of an individual function call.  This is consistent with the
objective of replacing Prolog procedures containing \textit{cuts}.  
PrologPF does not
attempt to exploit all the parallelism available in the non-deterministic but
complete evaluation of functions treated as general equational theories using
algorithms such as lazy narrowing \cite{CL91}.

While PrologPF provides a consistent environment for
higher-order functional
programming, the language has the same syntax (with the definition of some
additional operators) as normal Prolog.  Thus a PrologPF program can be
read by a standard Prolog compiler to produce a program in which all
function applications are treated as irreducible Prolog terms.

By careful selection of the specially treated operators, the functional
syntax of PrologPF will be familiar to users of Standard ML.


\subsection{Implementation Goals}

\begin{enumerate}

\item{To  be compatible with the Delphi principle, functional reduction
  must be deterministic}
\item{The capabilities of the functional component of PrologPF should
  minimise the requirement for \textit{cut} in the body of Prolog
  rules}
\item{The syntax should allow functional algorithms to be clearly expressed,
  with support for Prolog terms and variables including those representing
  functions, i.e. higher-order functions should be supported}
\item{The syntax and semantics of PrologPF should facilitate the
  straightforward use of functions within Prolog rules, and permit
  deterministic calls to Prolog procedures from within functions}

\end{enumerate}

%%%%%%%%%%%%%%%%%%%%%%%%%%%%%%%%%%%%%%%%%%%%
\section{Other Functional Logic Languages} %
%%%%%%%%%%%%%%%%%%%%%%%%%%%%%%%%%%%%%%%%%%%%

A more detailed review of the many approaches is given
in Chapter \ref{background}
and in \cite{BL86, Han94}.

While there seems to be no other implementation of a \textit{parallel}
functional logic language, there are many implementations of sequential
languages which support functional and logic programming.
They can be
broadly categorised as follows:
\begin{description}
\item [External functions callable from Prolog] {
  \cite{BM88,MBB+93} }

\item [Logic support provided within a functional programming system.] {
  These languages, such as LogLisp \cite{RS82}...}

\item [Logical relations treated as set-based functions] {
  e.g. \cite{CSA87}, Escher \cite{Llo94}..}

\item [Logic with equations and narrowing.] {
  \cite{DvH87}, ALF \cite{Han92}, Babel \cite{MNRA92}, K-LEAF \cite{BCM88},
  SLOG \cite{Fri85} }

\item [Callable function extensions to Prolog.] {
  Such as \textit{call/N} (cite Mercury) and \textit{apply/3} (cite Naish)
  }

\end{description}

%%%%%%%%%%%%%%%%%%%%%%%%%%%%%%%%%%%%%%%%%%%%%%%%%%%%%%%%%%
\section{Function definition: the \texttt{fun} relation} %
%%%%%%%%%%%%%%%%%%%%%%%%%%%%%%%%%%%%%%%%%%%%%%%%%%%%%%%%%%

\label{definition}

% extension of '=' relation
% guarded definitions can be 'A = B :- C'
% '=' redefinition:
%       less portable,
%       most flexible,
%       guards provide non-deterministic access to Prolog relations
%       no constraints on equality definitions implied, e.g.
%           left-linearity
%           '=' can represent bidirectional rewrite rules
% 'fun' relation:          
%       fun definition explicit, so determinism explicit
%       function namespace separate from relation namespace
%       syntax familiar to functional programmers
%       body of 'fun' relation can be compiled into efficient deterministic
%           machine code.
%       syntactic suppoert for currying in definition same as in application


\subsection{A PrologPF example}
%%%%%%%%%%%%

Before reviewing the syntax and semantics of PrologPF functions in detail with
comparison to other approaches, the following examples of the factorial and
append functions in PrologPF may place the alternatives in context.

Firstly, the factorial function:
\begin{verbatim}
fun fact(1) = 1;
    fact(N) = N * fact(N-1).
\end{verbatim}
or equally (see \ref{alt_if}):
\begin{verbatim}
fun fact(N) = if (N = 1)
              then 1
              else N * fact(N-1).
\end{verbatim}
The \texttt{append} function can be defined thus:
\begin{verbatim}
fun append(    [],Y) = Y;
    append([X|Xs],Y) = [X|append(Xs,Y)].
\end{verbatim}

\subsection{The PrologPF approach}
%%%%%%%%%%%%

Functions are defined in PrologPf with the special relation
\texttt{fun/1}, which is defined as a Prolog
prefix operator of low precedence (\texttt{op(1200,fx,fun)}).

Function definition in PrologPF also uses the \texttt{=} and \texttt{;}
operators but the standard Prolog precedence has been maintained.

The syntax supported is shown in table \ref{syntax:fun}

\begin{table}[htbp]
{\small
\begin{tabular}{|l r l|}
\hline
 & &\\[2mm]
Function\_{}Definition & ::= & \texttt{\textbf{fun}}
                               Alternate\_{}Definitions
                               \texttt{\textbf{.}}\\[4mm]

Alternate\_{}Definitions & ::= & Fun\_{}Equality\\
                      & $|$ & Fun\_{}Equality \texttt{\textbf{;}}
                              Alternate\_{}Definitions\\[4mm]

Fun\_{}Equality  & ::= & Fun\_{}Head \texttt{\textbf{=}} PrologPF\_{}Term\\[4mm]

Fun\_{}Head      & ::= & Prolog\_{}Atom \texttt{\textbf{(}} Args\ldots \texttt{\textbf{)}}\\
                 &  $|$  & Prolog\_{}Atom \texttt{\textbf{@ [}} Args\ldots
                            \texttt{\textbf{]}}\\
                 &  $|$  & Prolog\_{}Atom \texttt{\textbf{@ []}}\\[4mm]

Args             & ::= & Prolog\_{}Term\\
                 &   $|$ & Prolog\_{}Term \texttt{\textbf{,}} Args\\[4mm]

PrologPF\_{}Term & ::= & Prolog\_{}Term\\
                 & $|$ & Function\_{}Application\\[4mm]
 
\hline
\end{tabular}
}
\caption{Syntax: Function Definition with the \texttt{fun} Relation}
\label{syntax:fun}
\end{table}

In PrologPF, the underlying Delphi Machine has been extended to support \textit{cut}
(see Chapter \ref{cut}), and this support is exploited to implement deterministic 
functional reduction.

Each \texttt{fun} relation is transformed through a process of \textit{flattening}
(\cite{CF93}) into a deterministic procedure, with the actual arguments being
matched against the formal parameters until a successful unification is made, at which
point the choice of equality rule is committed and the reduction continuing with the
term on the right-hand-side.

Thus the selection of the appropriate equality rule is top-down, and the rewrite
is strictly left-to-right.

The equality is required to be \textit{constructor-based}, that is the terms in the
function head must not themselves contain any defined functions. 
This requirement is also described as \textit{head normal form} \cite{HAK+97}.
The syntax of the
formal parameters is given in table \ref{syntax:fun} as Prolog\_{}Term, i.e. a
standard Prolog term not including the application of any defined functions.

While the operational semantics of function evaluation in PrologPF 
have most in common with
languages such as Standard ML (\cite{Pau91}), the argument matching process is
replaced with Prolog's \textit{unification}.  Argument unification in PrologPF thus
differs from the matching in functional languages such as ML in two significant
ways:
\begin{enumerate}
\item{There is no requirement for left-linearity
  in the equality rules, i.e. variables can be repeated in the function head.  The
  functional component of PrologPF, like the underlying Prolog, has no occurs check.
  As with Prolog, it is the programmer's responsibility to avoid actual parameters which
  would cause the unification algorithm to loop, as with the goal \texttt{:- Y = a(Y)}.
}
\item{Partially instantantiated data structures (i.e. terms
containing variables) can be passed as arguments and returned as results.  This
means that, for example, difference lists can be supported and that a list of 
variables can be appended to another.}
\end{enumerate}

The Prolog atom used to name a defined function denotes a function of fixed arity, set
by the number of formal parameters given in the \texttt{fun} relation.  Alternative
definition of functions using the same name but a differing number of parameters is
flagged as an error by the PrologPF compiler.  This approach clearly differs from the
Prolog style where a relation name can be considered a combination of the naming
atom and the arity (as in \texttt{foo/2}), but is essential to permit currying within
the standard Prolog syntax.

\subsection{Alternative approaches}
%%%%%%%%%%%%

\subsubsection{Deterministic relations in Prolog}

Within Prolog, it is possible to define deterministic relations which then
can be treated as functions:

\begin{verbatim}
fact(1,1).
fact(N,F) :- N > 1, N1 is N - 1, fact(N1,F1), F is N * F1.
\end{verbatim}

In general, however, determinism inference is an undecidable problem, at
least dependent upon the solution of the halting problem:

\begin{verbatim}
foo(X,Y) :- complicated(X,Y).
foo(X,X).
\end{verbatim}

\texttt{foo/2} can have more than one solution only if
\texttt{complicated/2} can succeed.

In many cases, the programmer uses \textit{cut} within the Prolog
program to ensure determinacy of an otherwise non-deterministic
relation. For example:

\begin{verbatim}
fact(1,1) :- !.
fact(N,F) :- N1 is N - 1, fact(N1,F1), F is N * F1.
\end{verbatim}

However, the presence of \textit{cut} is not enough to
guarantee determinacy, as in the following example:

\begin{verbatim}
a(a).
a(b) :- !.
a(c).
\end{verbatim}

The query \texttt{:-a(X).} has the multiple solutions
\texttt{X=a, X=b}.

Deterministic reduction is essential for the successful
support of functions on the Delphi Machine (see Chapter
\ref{cut}), so the use of un-annotated Prolog relations to define
functions would introduce a significant possibility of error.

\subsubsection{Mercury}

In the Mercury system, each procedure is annotated with determinism
information \cite{HCSR95}. The syntax of Prolog relation definition
permits the use of relations and functions in multiple \textit{modes},
i.e. differing arguments being instantiated at the time of the call,
with others expected as results.  Mercury functions are thus annotated
with determinism information for each mode.
For example:
\begin{verbatim}
:- pred factorial(int, int).
:- mode factorial(in,out) is det.

factorial(N, F) :-
    ( N =< 0 ->
        F = 1
    ;
        N1 is N - 1,
        factorial(N1, F1),
        F is F1 * N
    ).
\end{verbatim}
Note that the mode information defines \texttt{factorial} to be
\texttt{det}, i.e. deterministic, while the relational style of 
definition is retained.
The Mercury compiler checks the supplied determinism information by
analysis of the code.  In this example the alternative representation of
the function shown below would be inferred to be non-deterministic through
limitations in the compiler's analysis of mutually exclusive conditions,
so the earlier if-then-else form must be used:
\begin{verbatim}
factorial(0, 1).
factorial(N, F) :-
    N > 0,
    N1 is N - 1,
    factorial(N1, F1),
    F is F1 * N.
\end{verbatim}
The use of Mercury's determinism and type inferencing techniques have
potential for exploitation on the Delphi Machine.  In PrologPF all
functions are, to use Mercury terminology, semi-deterministic.  That is
they can succeed once or fail.  Non-determistic modes of functions are
not required, and the syntax of function definition and application can
be considerably simplified and optimized for the deterministic use.

\subsubsection{Curry}
%%%%%%%%%%%%%%%

The logic capabilities of the language Curry \cite{HAK+97} are provided
through the support for \textit{non-deterministic functions}, and the
function definition syntax supports this:
\begin{verbatim}
f :: Int -> Int
f 1 = 10
f 2 = 20
f 2 = 30
\end{verbatim}
The language is typed, with \texttt{f} defined as
\textit{int $\rightarrow$ int} above.  The call \texttt{f 2} will
produce the multiple results \texttt{20} and \texttt{30}.
The left-hand-sides of the functional equality definitions can be
defined with conditional guards, such that the definitions are
referred to as \texttt{conditional equations} where the conditions are
constraints which must be solved in order for the equation to be applied.
This form is used in the definition of \texttt{factorial}:
\begin{verbatim}
factorial :: Int -> Int
factorial 1 = 1
factorial n | n > 1 = n * factorial (n - 1)
\end{verbatim}
The constraint \texttt{n $>$ 1} is added to the second equality defining
the factorial function to ensure deterministic evaluation of 
\texttt{factorial 1} which would otherwise match the rhs of both rules.
To ensure deterministic execution of a function in Curry, the defining
equations
must be checked to ensure that the conditions are not simultaneously
satisfiable \cite{MNRA92}, and no new variables can be introduced
in the equations' right-hand-sides.

The condition constraint in Curry can also be a boolean function
expression, as an abbreviation for the rule \texttt{<bool\_{}expr>=True}.
This is similar to the treatment of function applications
in relation positions in PrologPF, discussed in Section \ref{bool}.

\subsubsection{External procedures}

The functions can be defined in a language other than Prolog, and
called as external procedures.  Many existing implementations of Prolog
support this capability, and effort has been made to formalise the
approach (\cite{BM88, MBB+93, Bon92}).  These systems do not support 
higher-order programming.

\subsubsection{Logic programming with equality}

A more general solution is to define functions in terms of a set of
equalities \cite{Han94, Nai91},
extending Prolog's '\texttt{=}' relation, with conditional
support provided in the form of \textit{guards}.  For example:

\begin{verbatim}
fact(1) = 1.
fact(N) = N * fact(N-1) :- N \== 1.
\end{verbatim}

The use of guards (in this example \texttt{N $\backslash$== 1})
provides access to Prolog relations, including those
with multiple solutions.  The use of the equality relation itself imposes
no constraints on the form of the definition, permitting for example

\texttt{append(X,append(Y,Z)) = append(append(X,Y),Z).}\\

This is useful if a most general equation solving procedure is to
be used, with non-deterministic selection of rewrite rules and
of terms for reduction, and right-to-left as well as left-to-right
application of each equality rule.

The non-deterministic solution of equations would provide interesting
opportunities for the application of the Delphi principle to the
extended proof tree.  However, the research in this dissertation
ensures the functional reduction process is deterministic such that
the parallelized program has the efficiency associated with
direct execution
of compiled machine code.

%%%%%%%%%%%%%%%%%%%%%%%%%%%%%%%%%%%%%%%%%%%%%%%%%%%%%%%%%
\section{Function application: the \texttt{@} operator} %
%%%%%%%%%%%%%%%%%%%%%%%%%%%%%%%%%%%%%%%%%%%%%%%%%%%%%%%%%

\subsection{Extending Prolog for explicit function application}
%%%%%%%%%%%

The standard syntax for Prolog terms is supported, with special meaning 
applied to a new operator \texttt{@} (defined in PrologPF as
\texttt{op(600,yfx,@)}).  The presence of the operator in a PrologPF
term indicates that the normal unification step should be preceded by
functional evaluation.  For example, in the goal for the relation \texttt{=}:

\texttt{:- Z = foo @ [a].}

the term \texttt{foo @ [a]} should be evaluated
before the terms \texttt{Z} and the result of
\texttt{foo @ [a]} are unified with
the arguments of the \texttt{=} relation.

If \texttt{foo} is a \textit{defined function} 
(i.e. defined with the \texttt{fun} relation described in 
section \ref{definition}), then the rewrite rules specified
in the associated \texttt{fun} relation are used for the reduction.
Otherwise \texttt{foo} is a \textit{constructor} and the term is
irreducible.

For nested \texttt{@} terms, function evaluation is \textit{strict},
i.e. innermost arguments are evaluated first.
For example in:

\texttt{:- Z = foo @ [goo @ [a], hoo @ [b]].}

the terms \texttt{goo @ [a]} and \texttt{hoo @ [b]} will be evaluated
before the results are used in the evaluation of \texttt{foo} with
those arguments.  The evaluation of argument terms takes place
left-to-right.  Evaluation ordering is significant in PrologPF because
the usual functional programming one-way \textit{matching} is
replaced with \textit{unification}, and variable arguments are permitted.

\todo{example of argument unification needed here?}

The full \texttt{@} syntax is given in table \ref{syntax:application}.

\begin{table}[htbp]
{\small
\begin{tabular}{| l r l |}
\hline
 & & \\[2mm]
Function\_{}Application & ::= & Function\_{}Term \texttt{\textbf{@ [}} Args\ldots \texttt{\textbf{]}}\\
 & $|$ & Function\_{}Term \texttt{\textbf{@ []}}\\
 & $|$ & Defined\_{}Atom \texttt{\textbf{(}} Args\ldots \texttt{\textbf{)}}\\[4mm]
Function\_{}Term & ::= & Defined\_{}Atom\\
 & $|$ & Variable\\
 & $|$ & Lambda\_{}Expression\\
 & $|$ & Function\_{}Application\\[4mm]
Lambda\_{}Expression & ::= & \texttt{\textbf{lambda([}} Formal\_{}Args\ldots \texttt{\textbf{] ,}}
                                PrologPF\_{}Term \texttt{\textbf{)}}\\
 & $|$ & \texttt{\textbf{lambda([],}} PrologPF\_{}Term \texttt{\textbf{)}}\\[4mm]
Formal\_{}Args\ldots & ::= & Prolog\_{}Term\\
 & $|$ & Prolog\_{}Term \texttt{\textbf{,}} Formal\_{}Args\ldots\\[4mm]
Args\ldots & ::= & PrologPF\_{}Term\\
 & $|$ & PrologPF\_{}Term \texttt{\textbf{,}} Args\ldots\\[4mm]
Defined\_{}Atom & ::= & Prolog\_{}Atom defined in earlier \texttt{fun} clause\\
\hline
\end{tabular}
}
\caption{Syntax: Function Application with the \texttt{@} Operator}
\label{syntax:application}
\end{table}

Note that a function is always applied to a \textbf{list} of arguments,
so terms such as
\texttt{foo @ a} or \texttt{foo @ X}
do \textbf{not} denote function application (the correct syntax would be
\texttt{foo @ [a]} and \texttt{foo @ [X]}).

A function \texttt{foo} can be
defined with no arguments, and the reduction of that function can be made
explicit with \texttt{foo @ []}.  This use of \textit{nil} is similar to
the value \textit{unit} in Standard ML, and is useful where function
abstractions are used to emulate laziness, as in the example with infinite
lists in Chapter \ref{case_fcs}.  Nil argument functions are
discussed further in section
\ref{unit}.

\subsection{Function application: syntactic sugaring}
%%%%%%%%%%%
It should be noted that in PrologPF the term:

\texttt{foo(a,b)}

in which \texttt{foo} is a defined function, is semantically equivalent to:

\texttt{foo @ [a,b]}

This allows the most convenient syntax for function application to be used
within Prolog programs and allows consistent treatment of constructors and
functions. For example, the solution of the goal:

\texttt{:- Z = foo(goo(a),hoo(b)).}

can involve functional reduction of any of \texttt{foo}, \texttt{goo}, or
\texttt{hoo}.  With
\texttt{fun goo(X) = gg.} and \texttt{fun hoo(X) = hh.} then the goal
will succeed with the single solution \texttt{Z = foo(gg,hh)}.

This consistent treatment of constructors and functions can be seen in the
definition of a \texttt{wrap}
function which maps a list to a similar list with each element wrapped with
the constructor \texttt{envelope}:
\begin{verbatim}
fun wrap([])    = [];
    wrap([X|T]) = [envelope(X)|wrap(T)].
\end{verbatim}
\todo{cite Clocksin's new book}

%%%%%%%%%%%%%%%%%%%%%%%%%%%%%%%%%%%%%%%%%%%%%%%
\section{Higher-Order Functions and Currying} %
%%%%%%%%%%%%%%%%%%%%%%%%%%%%%%%%%%%%%%%%%%%%%%%
\label{higher-order}

A goal of the PrologPF system is to support functions as first-class data
items in the extended Prolog semantics, and to permit a syntax which 
facilitates the straightforward creation and application of function closures.

The approach in PrologPF owes much to Standard ML, with support for
nameless functions as lambda-expressions and the creation of closures
via currying \cite{Cur30, Sch24}.

\subsection{Lambda-expressions}
%%%%%%%%%%%

Nameless functions are created in PrologPF using the special constructor
\texttt{lambda/2}.  The syntax is given in table \ref{syntax:application}.

An example of a goal using a lambda expression representing the increment
function is:

\texttt{:- Z = lambda([X],X+1) @ [6].}

returning the single solution \texttt{Z = 7}.

As with defined functions in PrologPF,
the evaluation of the function term proceeds
with the unification of the actual parameter (in this example
\texttt{6}) with the argument of the lambda expression (\texttt{X}).
The instantiated second argument of the \texttt{lambda} term is then evaluated
to produce the final result.

Unlike standard Prolog, the scope of the formal arguments of the lambda
expression (\texttt{X} in the example above) is limited to that expression.
This ensures the correct operation of goals such as:\\
\centerline{\texttt{:- Y = lambda([X],X+1) @ [6], Z = lambda([X],X*2) @ [7].}}

Lambda terms can be defined to take \textbf{no} arguments, providing a
mechanism to delay the evaluation of the expression given as the second
argument.  For example:\\
\texttt{Z = lambda([],f(100))}\\
The expression \texttt{f(100)} will not be evaluated until a subsequent application
\texttt{Z @ []}.  This use of \textit{nil} arguments is discussed further in
Section \ref{unit}.

\subsection{Currying}
%%%%%%%%%%%

The support for currying in PrologPF ensures that the following equivalence holds
true:\\
\texttt{foo @ [a] @ [b] @ [c]} $\equiv$ \texttt{foo @ [a,b,c]}

The arity of a defined function is fixed in the \texttt{fun} relation
(section \ref{definition}). Any alternate definition using the same function
name but with a differing number of formal parameters is flagged by PrologPF
as an error.

This means the PrologPF compiler can generate appropriate code to return a lambda
expression where a function is called with fewer arguments than appear in the
\texttt{fun} definition. The earlier definition of the operator \texttt{@} was shown to be
left-associative (the 'yfx' in \texttt{op(600,yfx,@}).

These capabilities combine to provide the flexible support for
higher-order abstraction through the partial application of functions, known as currying.
For example, if a function \texttt{foo} is defined
with 3 arguments as in \texttt{fun foo(X,Y,Z) = X+Y+Z.}, then
(using symbol $\leadsto$ to represent 'evaluates to')

\texttt{foo @ [a]} $\leadsto$ \texttt{lambda([Y,Z],foo(a,Y,Z))}\\
$\Longrightarrow$\\
\begin{tabular}{l l l}
\texttt{foo @ [a] @ [b] @ [c]}
 & $\equiv$   & \texttt{((foo @ [a]) @ [b]) @ [c]}\\
 & $\leadsto$ & \texttt{(lambda([Y,Z],foo(a,Y,Z)) @ [b]) @ [c]}\\
 & $\leadsto$ & \texttt{lambda([Z],foo(a,b,Z)) @ [c]}\\
 & $\leadsto$ & \texttt{foo(a,b,c)}\\
 & $\equiv$   & \texttt{foo @ [a,b,c]}\\
\end{tabular}

The explicit use of the \texttt{@} operator and the use of currying permit the
straightforward definition and application of functions such a \texttt{map}:
\begin{verbatim}
fun map(F,[]) = [];                        % map definition
    map(F,[X|Xs]) = [F @ [X]|map(F,Xs)].

:- Z = map(+1,[10,20,30]).                 % curried +

:- Inc = map(+1), Z = Inc @ [[10,20,30]].  % curried map, +
\end{verbatim}
Each query succeeds with the single solution for \texttt{Z = [11,21,31]}.

%%%%%%%%%%%%%%%%%%%%%%%%%%%%%%%%%%%%%%%%%%%%%%%%%%%%%%
\section{Special Treatment of \texttt{if-then-else}} %
%%%%%%%%%%%%%%%%%%%%%%%%%%%%%%%%%%%%%%%%%%%%%%%%%%%%%%
\label{if-then-else}

PrologPF includes a predefined function \texttt{if} to provide conditional
evaluation of alternative expressions.  The systematic eager evaluation in
PrologPF precludes the definition of \texttt{if} as a normal PrologPF function
with three arguments:
\begin{verbatim}
fun if(true, A,B) = A;
    if(false,A,B) = B.
\end{verbatim}
As the argument evaluation semantics of PrologPF are eager,
in an expression such as \texttt{if(Z=0, 1, 100/Z)} all three arguments
would be evaluated before the application of \texttt{if}, producing a
possible
run-time arithmetic error during the attempted evaluation of \texttt{100/Z}.

To provide more useful behaviour, \texttt{if} is treated as a predefined
function with exceptional semantics.  The special treatment is unique
to \texttt{if}:
\begin{enumerate}
\item{The evaluation of the alternative expressions is delayed until
      \textbf{after} the condition has determined which of the
      two alternatives should be evaluated.  Only \textbf{one} of the
      two alternatives will then be evaluated.}
\item{The condition term is treated as a Prolog \textbf{goal}, rather
      than a boolean-valued reducible expression}
\end{enumerate}

\subsection{Syntax}
%%%%%%%%%%%

The syntax for the conditional \texttt{if} expression is given in
table \ref{syntax:if}.

\begin{table}
{\small
\begin{tabular}{| l r l |}
\hline
 & & \\
If\_{}Expression & ::= & \texttt{\textbf{if(}}PrologPF\_{}Term$_1$\texttt{\textbf{,}}
                                              PrologPF\_{}Term$_2$\texttt{\textbf{,}}
                                              PrologPF\_{}Term$_3$\texttt{\textbf{)}}\\
 & $|$ &  \texttt{\textbf{if}} PrologPF\_{}Term$_1$ \texttt{\textbf{then}}
                               PrologPF\_{}Term$_2$ \texttt{\textbf{else}}
                               PrologPF\_{}Term$_3$\\
 & $|$ &  \texttt{\textbf{if}} PrologPF\_{}Term$_1$ \texttt{\textbf{then}}
                               PrologPF\_{}Term$_2$\\[4mm]
\hline
\end{tabular}
}
\caption{Syntax: \texttt{if}}
\label{syntax:if}
\end{table}

The use of the predefined operators \texttt{if}, \texttt{then} and \texttt{else}
is permitted to reduce the use of brackets and allow a syntax similar to that
of languages such as Standard ML.  Where the \texttt{if-then-else} form is used,
the resultant expression is equivalent to the term\\
\texttt{if(Term$_1$,Term$_2$,Term$_3$)}.

To allow a convenient syntax without modifying the precedence of the standard
Prolog operators, the following precedences are used for \texttt{if},
\texttt{then} and \texttt{else}:
\begin{verbatim}
:- op(675,fx,if).     % 'if' is prefix
:- op(650,xfx,then).  % 'then' is infix
:- op(625,xfx,else).  % 'else' is infix
\end{verbatim}
The precedence of the predefined \texttt{if}, \texttt{then} and \texttt{else}
operators in PrologPF implies that:\\\
\texttt{if} Term$_1$ \texttt{then} Term$_2$ \texttt{else} Term$_3$\\
$\equiv$ \texttt{if (}Term$_1$ \texttt{then (}Term$_2$ \texttt{else} Term$_3$ \texttt{))}\\
$\equiv$ \texttt{if(}\texttt{then(}Term$_1$,\texttt{else(}Term$_2$,Term$_3$\texttt{)))}

The \texttt{else}-expression can be omitted, such that:\\
\texttt{if} Term$_1$ \texttt{then} Term$_2$ \texttt{~~}$\equiv$
\texttt{~~if} Term$_1$ \texttt{then} Term$_2$ \texttt{else fail}\\

The precedence of the \texttt{if-then-else} compound term has been set
higher than that of the Prolog's \texttt{=} and \texttt{;} operators
to minimise the need
for brackets in function definitions, and in goals of the form
\texttt{Z = if-expression}.  The compromise means that conditional
operators used in \texttt{if} conditions (i.e. Term$_1$) must
be bracketed, as must be nested \texttt{if} expressions. For example:
\begin{verbatim}
if (A < 20) then (if (A > 12)
                  then middle
                  else lower
                 )
            else upper
\end{verbatim}

\subsection{Evaluation}
%%%%%%%%%%%

Special code is generated in the call to \texttt{if} in the evaluation of
if-expressions.

\subsubsection{Defined evaluation ordering with \texttt{if}}

For any other arity/3 function call such as
\mbox{\texttt{foo(}Term$_1$\texttt{,}Term$_2$\texttt{,}Term$_3$\texttt{)}}
for defined function \texttt{foo}, code
of the following form would be generated:

[code to evaluate Term$_1$ with result as term X$_1$] \\[1mm]
[code to evaluate Term$_2$ with result as term X$_2$] \\[1mm]
[code to evaluate Term$_3$ with result as term X$_3$] \\[1mm]
functional evaluation of \texttt{foo(}X$_1$,X$_2$,X$_3$\texttt{)}\\

In the case of the special function \texttt{if} the eager evaluation of both
alternative expressions in terms such as 
\texttt{if (Z = 0) then 1 else 100/Z} would
not execute as intended for \texttt{Z = 0}, so consequently
code of the following form will be generated:

[code to find first solution of call(Term$_1$)
as relational goal (section \ref{if:cond})] \\[1mm]
$<$on success:$>$ [code to return result of evaluation of Term$_2$] \\[1mm]
$<$on failure:$>$ [code to return result of evaluation of Term$_3$] \\

PrologPF ensures that:
\begin{enumerate}
\item{The condition goal completes \textbf{before} the evaluation of the
  alternate expressions of the \texttt{if}-expression.}
\item{The condition goal succeeds with one solution or fails.}
\item{Only one of the alternate expressions will be evaluated, the \texttt{then}
  expression if the condition goal succeeds, or the \texttt{else} expression
  if it fails.}
\end{enumerate}

\subsubsection{\texttt{if} condition as relational goal}
\label{if:cond}

There is considerable advantage in giving functions within the combined
functional logic system access to the relations in the program and those in
the Prolog libraries.  The implementation chosen for the Delphi Machine
requires that the function evaluation be deterministic.  A successful
compromise has been achieved with:
\begin{enumerate}
\item{The \textbf{only} place a function in PrologPF can call a Prolog
  relation is in the condition of an \texttt{if}-expression}
\item{The call uses Prolog's normal search, but determinism is maintained with
  first-solution semantics}
\item{The acceptance of boolean functions as relational goals reintroduces
  functional terms as conditions (section \ref{bool})}
\end{enumerate}

An example showing how the Prolog library \texttt{append} relation can
be used to produce a similar (but deterministic) function would be:
\begin{verbatim}
fun append(X,Y) = if append(X,Y,Z) then Z.
\end{verbatim}
This example relies upon the following:
\begin{enumerate}
\item{The \texttt{if} semantics ensure the goal \texttt{append} produces a
  value for \texttt{Z} before the evaluation of the sub-expressions
  \texttt{Z} and \texttt{fail}.}
\item{The \texttt{if} semantics ensure that
  only one of the sub-expressions is
  evaluated, after the
  solution of the conditional goal.}
\item{The relation \texttt{append/3} and the function \texttt{append/2} are recognized
  as having different names (see section \ref{bool})}
\item{The missing \texttt{else}-expression is equivalent to \texttt{else fail}, so the
  definition is an abbreviation for:\\
  \centerline{\texttt{fun append(X,Y) = if append(X,Y,Z) then Z else fail.}}
  }
\item{The predefined function \texttt{fail} is available to produce
  function failure (section \ref{fail})}
\end{enumerate}

The use of relational goals as conditions, combined with Prolog's left-to-right
search rule,
leads to a Prolog syntax with semantics similar to the special operators in languages
such as Standard ML for \texttt{andalso} and \texttt{orelse}:

\begin{tabular}{l l l l}
Conjunction: & \texttt{(P,Q,R)} & $\equiv$ & \texttt{P andalso Q andalso R}\\
Disjunction: & \texttt{(P;Q;R)} & $\equiv$ & \texttt{P orelse Q orelse R}
\end{tabular}

For example, using the standard Prolog library relations $>$ and $<$:
\begin{verbatim}
fun account_status(Bal) = if (Bal > 0, Bal < 100)
                          then normal
                          else needs_attention.
\end{verbatim}

\todo{provides similarity with conditional guards while enforcing
determinism}
\todo{Prolog's A -$>$ B,C has similar deterministic condition A}

\subsection{Value declarations}
%%%%%%%%%%%
\label{val}

A value declaration gives an expression a \textit{name} within a
particular \textit{scope}.

The PrologPF support for \texttt{if} if ensures that the relational
condition is executed before the alternate expressions.  The
unifier of the free variables in the condition is thus valid for
the evaluation of the \texttt{then}-expression, which is only
evaluated if the condition has succeeded.  Thus the use of the
\texttt{=} relation in the condition of an
\texttt{if-then-else} expression can give a value a name, which will
be valid in the scope of the \texttt{then} sub-expression.

The use of the unification of the condition to support naming in this
way is convenient if a sub-expression is to be repeated within an
expression, as often occurs within an \texttt{if-then-else}.  An
example is in a definition of a \texttt{max} function to find
the highest integer in a list:
\begin{verbatim}
fun max([X])    = X;
    max([X|Xs]) = if (M = max(Xs))
                  then (if (X > M)
                        then X
                        else M
                       ).
\end{verbatim}
In the recursive case,  the condition goal \texttt{M = max(Xs)} results
in the evaluation of \texttt{max(Xs)} being unified with a new free
variable \texttt{M}, with the unifier \texttt{M/n} (where n is the
largest integer in \texttt{Xs}) being valid for the subsequent
evaluation of \texttt{if (X > M) then X else M}.

The use of \texttt{M} as a \textit{name} to represent the value
\texttt{max(Xs)} is equivalent to the repeated appearance of
the value in the \texttt{then} expression.  The \texttt{max} function
could equally be written:
\begin{verbatim}
fun max([X])    = X;
    max([X|Xs]) = if (X > max(Xs))
                  then X
                  else max(Xs).
\end{verbatim}
As these value declarations are using the standard \texttt{=}
relation in the condition, the method supports a convenient
technique for using functions that multiple results as
tuples.  This can be seen with the second of the
complementary functions
\texttt{zip} and \texttt{unzip}.  The function \texttt{zip}
takes two lists of equal length as arguments, and returns a
list of pairs \cite{HAK+97}:
\begin{verbatim}
fun zip([],[])         = [];
    zip([X|Xs],[Y|Ys]) = [(X,Y)|zip(Xs,Ys)].
\end{verbatim}
The complementary function \texttt{unzip} has a convenient
definition using a value declaration \cite{Pau91}:
\begin{verbatim}
fun unzip([])            = ([],[]);
    unzip([(X,Y)|Pairs]) = if ((Xs,Ys) = unzip(Pairs))
                           then ([X|Xs],[Y|Ys]).
\end{verbatim}
A version of \texttt{unzip} that did not use a value
declaration could be written using of auxiliary functions to
extract the elements of the tuple and repeating the \texttt{unzip(Pairs)}
sub-expression.  Alternatively, an auxiliary function could be
defined to add a pair of elements to pair of lists, as in:
\begin{verbatim}
fun addpair((X,Y),(Xs,Ys)) = ([X|Xs],[Y|Ys]).

fun unzip([])           = ([],[]);
    unzip([Pair|Pairs]) = addpair(Pair, unzip(Pairs)).
\end{verbatim}
However, the use of value declarations in \texttt{unzip} avoids the
use of auxiliary functions.

In the absence of common-expression elimination optimisations in the
PrologPF compiler, the use of value declarations results in a more efficient
object program.  In general, the use of names to represent expressions
that are complex or repeated can result in programs that are more
comprehensible.

\subsection{Alternate function definitions $\equiv$ \texttt{if}}
%%%%%%%%%%%
\label{alt_if}

In PrologPF, the function definition style using alternate argument patterns can
be shown to be equivalent to a single functional equality using the
predefined \texttt{if} function.

The following characteristics of the \texttt{if} function are
important in the equivalence:
\begin{itemize}
\item{The defined lazy conditional evaluation of the arguments to \texttt{if}}
\item{The call to the condition
  goal is defined to \textit{precede} the evaluation of one of the functional terms.
  Value declarations from unification of terms in the condition goal with local
  variables are therefore guaranteed to be bound in the scope of the subsequently
  evaluated dependent expression.}
\end{itemize}

With the example of the factorial function:
\begin{verbatim}
fun fact(1) = 1;
    fact(N) = N * fact(N-1).
\end{verbatim}
The equivalent if-then-else form is:
\begin{verbatim}
fun fact(Z) = if (Z=1)
              then 1
              else (if (Z=N)
                    then N * fact(N-1)
                   ).
\end{verbatim}
The general form of the translation is:
\begin{verbatim}
fun Function_name(Arg_pattern1, Arg_pattern2...) = Expression_1;
    Function_name(Arg_pattern3, Arg_pattern4...) = Expression_2...
\end{verbatim}
goes to:
\begin{verbatim}
fun Function_name(Var1,Var2...)
          = if (Var1 = Arg_pattern1, Var2 = Arg_pattern2...)
            then Expression_1
            else (if (Var1 = Arg_pattern3, Var2 = Arg_pattern4...)
                  then Expression_2
                  else ...
                 ).
\end{verbatim}
Value declarations in the relational goal of the \texttt{if} condition can be seen more
clearly with the transformation of the \texttt{append} function:
\begin{verbatim}
fun append(    [],Y) = Y;
    append([X|Xs],Y) = [X|append(Xs,Y)].
\end{verbatim}
goes to:
\begin{verbatim}
fun append(L,Y) = if (L=[])
                  then Y
                  else (if (L=[X|Xs])
                        then [X|append(Xs,Y)]
                       ).
\end{verbatim}

\subsection{Summary of PrologPF \texttt{if} semantics}
%%%%%%%%%%%%

The goal of the \texttt{if-then-else} implementation in PrologPF is to
provide useful conditional evaluation semantics, while supporting
deterministic access to relations.

With the expression 
\texttt{if} Term$_1$ \texttt{then} Term$_2$ \texttt{else} Term$_3$:
\begin{itemize}
\item{The conditional expression Term$_1$ is treated as a relational goal, either
  succeeding with a variable binding, or failing.}
\item{The depth-first, left-to-right search of standard Prolog is used to find
  a solution to Term$_1$, and the search is limited to finding the first solution.}
\item{The call to the conditional expression Term$_1$ completes before the evaluation
  of either Term$_2$ or Term$_3$.}
\item{If Term$_1$ succeeds then Term$_2$ is evaluated in the context of any bindings
  resulting from the solution of Term$_1$, and the result returned as the value of the
  \texttt{if}-expression.}
\item{If Term$_1$ fails then Term$_3$ is evaluated and returned as the result of the
  \texttt{if}-expression.}
\item{If the \texttt{else}-expression (\texttt{else} Term$_3$) is omitted, the semantics are
  the same as if an \texttt{else}-expression (\texttt{else fail}) were added.}
\end{itemize}

%%%%%%%%%%%%%%%%%%%%%%%%%%%%%%%%%%%%%%%%%%
\section{Boolean functions as relations} %
%%%%%%%%%%%%%%%%%%%%%%%%%%%%%%%%%%%%%%%%%%
\label{bool}

In summary, the following equivalence holds for functions used in
the position of relational goals:\\
 \texttt{?- foo(a).~~~} $\equiv$ \texttt{~~~?- foo(a) = true},
iff \texttt{foo} is a defined function of arity/1.

A function application term is permitted to appear in the position of
a relational goal, where it is treated as a call to the Prolog \texttt{=}
relation to unify the result of the function application with \texttt{true}.
This applies to the body of each rule and the condition of each \texttt{if}
expression.

For example, with a boolean function \texttt{prime(X)} returning true for
a prime argument and false otherwise, the goal:\\
\centerline{\texttt{?- p(X), prime(X), write(X).~~} $\equiv$
\texttt{~~?- p(X), prime(X) = true, write(X).}}

The explicit treatment of boolean functions as relations
in this way can be seen in the prototype
produced by Paulson and Smith \cite{PS91}.
The language Escher \cite{Llo94}
has \textbf{all} relations declared as boolean functions in this way.

Either the explicit \texttt{@} operator can be used to
denote the function application, or the Prolog compound term syntax can
be used.  In the latter syntax, the outermost functor of the goal will
only be recognized as a defined function if the number of actual parameters
matches the arity of the defined function of the same name.  The specification
of a reduced number of arguments in a curried application is not useful where
a boolean result is required.  A partial (i.e. curried) function application
would always return a higher-order result, such that:\\
\centerline{\texttt{<curried\_{}application> = true~~} $\equiv$ \texttt{~~fail}}

The requirement for the arities of the defined function and the actual use within
a goal
facilitates the conversion of library relations (such as \texttt{append}) into
functions and vice versa.  I.e. the functional definition of \texttt{append/2}
does not conflict with the relational definition \texttt{append/3}, and the
library relation can be used in the function definition:
\begin{verbatim}
fun append(X,Y) = if append(X,Y,Z) then Z.
\end{verbatim}
Equally, the deterministic functional version of append
given in section \ref{alt_if} could have been used for a
version of the library relation limited to deterministic modes:
\begin{verbatim}
append(X,Y,Z) :- Z = append(X,Y).
\end{verbatim}

To summarise the naming/arity issues arising from both currying and the
acceptance of boolean functions as relations:
\begin{enumerate}
\item{Each alternate equality statement in the definition of a function must
  have the same number of formal parameters, and this number is the arity of
  the function}
\item{A function can have the same name as a relation, but must not have
  the same arity}
\end{enumerate}
The first rule is to allow currying, the second to allow boolean functions as
goals.  The functional logic language Mercury has a similar rule to 2 above,
but in Mercury a function must not have an arity that is \textbf{one less} that a
relation of the same name.  The Mercury name/arity constraints are inconvenient,
as it is natural to define a function (such as \texttt{append/2}) to have an
arity one less that an equivalent relation (i.e. \texttt{append/3}).  PrologPF
exploits this capability to define functions representing deterministic modes
of many frequently-used
library relations such as \texttt{append} and \texttt{=..}.

In the design of PrologPF, a choice was made to introduce rule 2, rather than
the alternative that boolean functions as goals should require explicit use of 
the \texttt{@} operator.  The body of Prolog code converted for execution on
the PrologPF system has not yet included enough examples of relations with
multiple arities to confirm this design decision.

%%%%%%%%%%%%%%%%%%%%%%%%%%%%%%%%
\section{Failure of functions} %
%%%%%%%%%%%%%%%%%%%%%%%%%%%%%%%%
\label{fail}

The functional support in PrologPF is embedded within an environment of
relations which are expected to \texttt{succeed} (with an associated
variable binding) or \textit{fail}.  The treatment of function applications as
relation argument terms associates every application with an underlying relation,
for example in:\\
\centerline{\texttt{?- Z = fact(5).}}\\
the function application of \texttt{fact} is as an argument of the
relation \texttt{=}.

\subsection{Functional failure $\Rightarrow$ Relation failure}
%%%%%%%%%%%%
\label{func_fail}

In PrologPF, function failure is supported through the provision of a
special term \texttt{fail}.
This mirrors the standard Prolog relation \texttt{fail}, which
always fails.
\begin{enumerate}
\item{The evaluation of the term \texttt{fail} within an expression
  produces no value but always fails.}
\item{A function application fails if evaluation of a subexpression
  in the right-hand-side of the associated definition fails.}
\item{A relation fails if the evaluation of a functional argument
  fails.}
\end{enumerate}
The use of \texttt{fail} within a function definition can be seen
in the \texttt{lookup} function, which returns a value associated
with a key in a list of paired key-value terms:
\begin{verbatim}
fun lookup(_,[])                = fail;
    lookup(Key,[(Key,Value)|_]) = Value;
    lookup(Key,[_|T])           = lookup(Key,T).
\end{verbatim}
The function might be used in a program such as:
\begin{verbatim}
a(a).
a(b).
a(c).

?- a(X), write(lookup(X,[(a,1),(c,3),(e,5)])).
\end{verbatim}
The subgoal \texttt{a(X)} produces values \texttt{a},
\texttt{b} and \texttt{c} for \texttt{X}, calling \texttt{write}
with the value of the \texttt{lookup} application.  As the
key-value list argument contains no entry for \texttt{b}, the
application will fail for that argument value.  Backtracking will
take place as in standard Prolog, such that \texttt{write} will
display the values \texttt{1} and \texttt{3} from the successful
application of \texttt{lookup} with \texttt{a} and \texttt{c}.

\subsection{Function \texttt{fail} as an exception}
%%%%%%%%%%%%

Within the function evaluation, the semantics of \texttt{fail} are
those of an uncaught \textit{exception} \todo{cite exceptions}.
The exception can be considered to be caught at the point
immediately preceding the unification of the term with the
corresponding argument of the relation, where it causes that relation
to fail.

The general support for exceptions would be consistent with the rest of the
functional support in PrologPF as
\begin{enumerate}
\item{Function evaluation in PrologPF is \textit{eager}, so 
  the evaluation of the expression term to be raised as an
  exception can occur \textbf{before} the exception is raised
  and the \textbf{value} of the expression returned as the
  exception value.  A lazy functional language with 
  \textit{call-by-need} argument evaluation semantics would require
  special treatment of the expressions given to the \texttt{throw}
  function.}
\item{PrologPF permits partially defined functions (where some
  legal actual argument patterns have no matching left-hand-side in
  the function definition) and function failure.
  A more general support for exceptions can be provided for
  which these are special cases.
  }
\end{enumerate}

If, as in Standard ML \cite{MTH90}, a general support for
exceptions were provided though the use of \texttt{raise} and
\texttt{handle} operators, then the use of \texttt{fail} within
PrologPF could be shown to be equivalent to the limited use of 
those exceptions:\\
\begin{tabular}{l l l}
\texttt{fail} in PrologPF      &$\equiv$ &\texttt{raise Fail}\\
                               &         &with declared \texttt{exception Fail}\\[4mm]
With relation $R$, argument    &         &\\
expressions $e_1,...,e_n$, and &         & \\
goal $R(e_1,\ldots,e_n)$       &$\equiv$ &$e_i$ \texttt{handle Fail} $\Rightarrow$\\
                               &         &~~~~ensure \textit{failure} of $R$\\
                               &         &at each argument $e_i, i=1\ldots n$\\
\end{tabular}

\subsubsection{A proposal for more general exception support in PrologPF}
%%%%%%%%%%%%%%%

Standard Prolog \cite{DEC96} has support for exceptions at the
level of relations with the predefined \texttt{catch} and \texttt{throw}
metalogical operators.  An exception is generally referred to as a 
\texttt{Ball}.

The format for the use of \texttt{throw} is:\\
\centerline{\texttt{throw(Ball)}}\\
where \texttt{Ball} is any Prolog term to be propagated as an exception.
Similarly the format for the use of \texttt{catch} is:\\
\centerline{\texttt{catch(Goal,Ball,Handler)}}\\
where:\\
\texttt{Goal} is a Prolog relational goal potentially containing
\texttt{throw} subgoals,\\
\texttt{Ball} is a term to be unified with the actual argument of any
\texttt{throw} operators encountered during execution of \texttt{Goal}, and\\
\texttt{Handler} is a subgoal to be called when an exception is
caught, i.e. successfully unified with \texttt{Ball}.\\
Often, \texttt{Ball} and \texttt{Handler} will contain common free
variables, as a means of propagating values from the \texttt{throw}.
The goal:\\
\centerline{\texttt{catch(throw(foo),X,write(X))}}\\
will have the effect of writing ''foo`` to the output, with the
execution proceeding as follows:
\begin{enumerate}
\item{\texttt{catch} calls the subgoal given as its first argument,
  namely \texttt{throw(foo)}.}
\item{The subgoal \texttt{throw(foo)} throws (raises) the ball (exception)
  \texttt{foo}.}
\item{The ball \texttt{foo} 
  propagates to the level of the surrounding \texttt{catch}
  where it is unified with the second argument of the \texttt{catch}
  relation (\texttt{X}).  If this unification had failed, then the ball
  continues to propagate to any higher enclosing \texttt{catch} relation.}
\item{With the successful unification of \texttt{foo} with \texttt{X},
  the subgoal \texttt{write(X)}
  given as the third argument to \texttt{catch} is called.}
\item{\texttt{foo} is written to the ouput.}
\end{enumerate}
In the context of standard Prolog's \texttt{catch} and \texttt{throw},
the use of \texttt{fail} within defined functions in PrologPF can be 
treated as:\\

\begin{tabular}{l l l}
\texttt{fail} in PrologPF       &$\equiv$ &\texttt{if throw(fail) then \_{} else \_{}}\\[4mm]
A goal containing relation $R$, &         &\\
as in $\ldots,R,\ldots$         &$\equiv$ &$\ldots,$\texttt{catch(}$R$\texttt{,fail,fail)}$,\ldots$\\
\end{tabular}

Note the use of \texttt{if-then-else} to map the relational call to
\texttt{throw} into an expression.  The implicit \texttt{catch} which can
be considered to be wrapped around each relation call containing functional
arguments is shown to only handle one value of exception (\texttt{fail}).
The \texttt{catch} goal will then fail if this type of exception is
caught.

With this definition we arrive at the semantics for our use of
\texttt{fail} within functions as uncaught exceptions, leading to
failure of the associated relation.

The definition using \texttt{catch} and \texttt{throw}
could lead to the more flexible use of exceptions within
the functional component of PrologPF,  although the implementation to
date only permits the support for \texttt{fail}.

An improved support would:
\begin{itemize}
\item{Allow any term to be raised as an exception value within a
  defined function, for example \texttt{throw(foo)}.}
\item{Allow exceptions to be caught within the functions rather than
  propagating to the relational level.}
\item{Treat any uncaught exception from a functional evaluation as
  \texttt{fail}.}
\end{itemize}
The implementation would require the following:
\begin{itemize}
\item{The meta-relation \texttt{throw} should be mapped to a
  similar function \texttt{throw/1}, 
  where an expression \texttt{throw(X)}
  would have the same meaning as\\
  \centerline{\texttt{if throw(exception(X)) then \_{} else \_{}}.}\\
  The
  definition would use the support in PrologPF for relations
  as if-conditions.  The use of anonymous variables as
  the alternate expressions is arbitrary, as the function
  \texttt{throw} would never return any value.
  Function could then use \texttt{throw}
  within any expression.}
\item{As with the metarelation \texttt{catch}, a functional
  equivalent would allow the handling of exceptions at any
  level of a nested functional expression, as in Standard
  ML.  The ML syntax is\\
  \centerline{$E$ \texttt{handle} $P_1 => E_1 | \ldots | P_n => E_n$}\\
  where $E$ is the expression which may possibly raise an
  exception, $P_i$ is an expression matching the exception and
  $E_i$ is the corresponding
  value to be returned instead of $E$.  The equivalent support
  in PrologPF would be by nested applications of a
  \texttt{catch} function, which would have the same capabilities as
  \texttt{catch(}$E$\texttt{,exception(}$P_i$\texttt{),}$E_i$\texttt{)} for
  each pattern $P_i$ for unification with the exception term thrown.
  }
\item{The implicit \texttt{catch} wrapper around each relation $R$
  would be\\
  \centerline{\texttt{catch(}$R$\texttt{,exception(X),fail)}.}\\
  This can
  be contrasted with the more limited form supporting \texttt{fail}
  given above.
  }
\end{itemize}


%%%%%%%%%%%%%%%%
\section{Unit} %
%%%%%%%%%%%%%%%%
\label{unit}

ML has a built-in type '\texttt{unit}' with only one member, namely
 '\texttt{()}'.
A function of intended arity zero will be defined of type
'\texttt{unit $\rightarrow \alpha$}', and 
the value of that function will be returned by the explicit application
of that function to '\texttt{()}'.

An example ML function definition of this type is:
\begin{verbatim}
>fun foo () = 22;
foo: unit -> int
>val a = foo ();
a = 22 : int
\end{verbatim}
In PrologPF, all functions are explicitly applied to a \textbf{list} of
actual arguments, using Prolog syntax for lists, and the application of
a function to \textbf{no} arguments can be explicit by using an empty
list (i.e. nil: '\texttt{[]}').  The application of a function to no
arguments simply returns that function, i.e.\\
\texttt{foo @ []} for defined function \texttt{foo} with arity 0 $\equiv$ evaluated \texttt{foo}\\
\texttt{bah @ []} for arity \texttt{bah} $>$ 0 is $\equiv$ \texttt{bah}\\
$\Rightarrow$ \texttt{bah @ [] @ [] @ []} $\equiv$ \texttt{bah}\\
$\Rightarrow$ \texttt{bah @ [] @ [] @ [X]} $\equiv$ \texttt{bah @ [X]}.

%%%%%%%%%%%%%%%%%%%%%%%%%%%%%%%%%%%%%%%%%%%%%%%%%%%%%%
\section{The Interaction of Functions and Relations} %
%%%%%%%%%%%%%%%%%%%%%%%%%%%%%%%%%%%%%%%%%%%%%%%%%%%%%%

In the combined functional and logic programming paradigm of PrologPF, most
effort has been placed in the design of the overlap between the use of
defined functions and realtional rules.  The resultant system allows the
exploitation of defined functions within rules and access to relations from
within functions in a straightforward way with clear semantics.

The interaction between the functional and logic elements of a PrologPF
program is limited to:
\begin{description}
\item[Function definition.]{The relation \texttt{fun} is given special meaning as
  declaring the ordered equational rewrite rules defining a named function.}
\item[Function application.]{The semantics of the actual argument
  terms of predicates has been extended to
  include the application of defined functions
  with the special operator \texttt{@}.  The functional reduction
  is defined to occur as a step preceding the unification of the resultant term
  with the predicate formal arguments.}
\item[Function failure.]{Function failure is defined, such that a goal with a failing
  function as an argument term is defined to fail.}
\item[Relation call from within functions.]{The condition term of the built-in
  PrologPF function \texttt{if} is defined to be a relational goal, with determinism
  ensured by one-solution call semantics.}
\item[Functions as goals.]{The non-curried application of a defined function as a goal is
  defined to be equivalent to the \texttt{=} goal with that application term and
  \texttt{true}.}
\item[Functions as first-class data items.]{A function abstraction returned as the
  result of a higher-order function or the user definition of a lambda-term can be
  unified with a logical variable for application within subsequent goals or sub-goals.}
\end{description}

%%%%%%%%%%%%%%%%%%%%%%%%%%%%%%%%%%
\section{Some PrologPF examples} %
%%%%%%%%%%%%%%%%%%%%%%%%%%%%%%%%%%

A comprehensive review of the application of PrologPF to both logic and functional
problems is given in Chapters \ref{case_fcs} and \ref{case_pttp}.

PrologPF examples of functions for factorial, append, map, and max 
have been given in
the preceding sections, and are repeated here for clarity:
\begin{verbatim}
fun fact(1) = 1;
    fact(N) = N * fact(N-1).

fun append(    [],Y) = Y;
    append([X|Xs],Y) = [X|append(Xs,Y)].

fun map(F,[]) = [];
    map(F,[X|Xs]) = [F @ [X]|map(F,Xs).

fun max([X])    = X;
    max([X|Xs]) = if (M = max(Xs))
                  then (if (X > M)
                        then X
                        else M
                       ).
\end{verbatim}

\subsection{Undergraduate Prolog exam question attempt}
%%%%%%%%%%%%

An interesting example of functional logic syntax could be seen in an
attempt by an undergraduate\footnote{Prolog Programming for AI, Part 1B,
Cambridge University 1997} to write a relation \texttt{remhigh/2} in which
the first argument is a list of integers, and the second is the same list
excluding the highest element.  The undergraduate wrote:
\begin{verbatim}
%%%% remhigh:    L2 is list L1 excluding highest member of L1

remhigh(L1,L2) :- remove( max(L1), L1, L2).

%%%% remove(Element, List, Remainder_list) :
%%%%     Remainder_list is List excluding Element

remove(N, [N|T], T).
remove(N, [H|T], [H|T1]) :- N \== H, remove(N, T, T1).
\end{verbatim}
From the definition of \texttt{remhigh} it can be seen that the student
expected a functional support that is not present in Prolog.  The student
is also suggesting a natural syntax.

The above attempt would be correct in PrologPF with the
definition of \texttt{max} given above in section \ref{if-then-else}.

\subsection{Lazy lists}
%%%%%%%%%%%%

This example is extended and reviewed in more detail in
chapter \ref{case_fcs}, where infinite streams of primes are
created.  Here we will show the use of the higher-order features of 
PrologPF to represent infinite lists.

Infinite lists in this program will be represented by constructor
terms of the form:\\
\centerline{\texttt{item(Head,Tail)}}\\
where \texttt{Head} is the value at the head of the list and \texttt{Tail}
is a \textbf{function} of arity zero which returns the tail of the list.  The empty
list can be represented by a constructor such as \texttt{empty}.
The functions to extract the components of a list are:
\begin{verbatim}
fun head(empty)     = fail;
    head(item(X,_)) = X.

fun tail(empty)     = fail;
    tail(item(_,F)) = F@[].
\end{verbatim}
A function to create the infinite list of natural numbers is:\\
\centerline{\texttt{fun make\_{}nats(N) = item(N,lambda([],make\_{}nats(N+1))).}}

The application \texttt{make\_{}nats(N)} can now be used to represent an
infinite list the natural numbers starting from N.

A goal such as \texttt{?- Z = head(tail(tail(make\_{}nats(1)))).} will
return the expected solution \texttt{Z = 3}.

With this representation of infinite lists, a version of the
higher-order function \texttt{map} can be defined in PrologPF:
\begin{verbatim}
fun imap(F,empty)     = empty;
    imap(F,item(X,T)) = item(F@[X], lambda([],imap(F,T@[]))).
\end{verbatim}
The function can be demonstrated in a goal such as\\
\centerline{\texttt{?- Z = head(tail(tail(imap(*2,make\_{}nats(1))))).}}\\
giving the solution \texttt{Z = 6}.

The \texttt{imap} function illustrates the combined use of
\textit{constructors} (\texttt{empty},\texttt{item}), \textit{higher-order
variables} (\texttt{F}), explicit application with \texttt{@}, implicit
application of \texttt{imap}, use of \texttt{lambda} expressions, and the
use of \textit{nil} to denote evaluation of an arity/0 function.
The example shows that the syntax facilitates the use of these capabilities
without obscure programming constructs.


%%%%%%%%%%%%%%%%%%%%%%%%%%%%%%%%%%%%%%%%%%%%%%%%%%%%%%%%%%%%%%%%
\section{Comparison of PrologPF with \texttt{call/N, apply/3}} %
%%%%%%%%%%%%%%%%%%%%%%%%%%%%%%%%%%%%%%%%%%%%%%%%%%%%%%%%%%%%%%%%

The semantics of the support for functions in PrologPF has most in
common with Naish's \texttt{apply/3} \cite{Nai96}, although he retains the
definition of functions as Prolog relations, and permits non-deterministic
evaluation.  Naish's definition of \texttt{apply/3} is designed as a more
capable replacement for the \texttt{call/N} extralogical predicate provided
in some Prologs and used as the basis for the higher-order functional
support in Mercury \cite{SHC95}.

Table \ref{call_apply} compares PrologPF with \texttt{call/N} and \texttt{apply/3}
using the examples from \cite{Nai96}.

\begin{table}[htbp]
{\footnotesize
\tt
\begin{tabular}{| l | l | l |}
\hline
\texttt{~~~~~~~call/N}     & \texttt{~~~~~~~apply/3}   &\texttt{~~~~~~~PrologPF}\\
\hline
& & \\
map(F,[],[]).              &map(F,[],[]).              &fun map(F,[])~~~~~= [];\\
map(F,[X|Xs],[Y|Ys]) :-    &map(F,[X|Xs],[Y|Ys]) :-    &~~~~map(F,[X|Xs]) = [F @ [X]|map(F,Xs)]\\
~~~~call(F,X,Y),           &~~~~apply(F,X,Y),          &\\
~~~~map(F,Xs,Ys).          &~~~~map(F,Xs,Ys).          &\\
%\hline
& & \\
filter(P,[],[]).           &filter(P,[],[]).           &fun filter(P,[])~~~~~= [];\\
filter(P,[X|Xs],Ys) :-     &filter(P,[X|Xs],Ys) :-     &~~~~filter(P,[X|Xs]) =\\ 
~~~~(call(P,X) ->          &~~~~(apply(P,X,true) ->    &~~~~~~~~if (P @ [X])\\
~~~~~~~Ys = [X|Z]          &~~~~~~~Ys = [X|Z]          &~~~~~~~~then [X|filter(P,Xs)]\\
~~~~;                      &~~~~;                      &~~~~~~~~else filter(P,Xs).\\
~~~~~~~Ys = Z              &~~~~~~~Ys = Z              & \\
~~~~),                     &~~~~)                      & \\
~~~~filter(P,Xs,Z).        &~~~~filter(P,Xs,Z).        & \\
%\hline
& & \\
foldr(F,B,[],B).           &foldr(F,B,[],B).           &fun foldr(F,B,[])~~~~~= B; \\
foldr(F,B,[X|Xs],R) :-     &foldr(F,B,[X|Xs],R) :-     &~~~~foldr(F,B,[X|Xs]) =\\
~~~~foldr(F,B,Xs,R1),      &~~~~foldr(F,B,Xs,R1),      &~~~~~~~~F @ [X,foldr(F,B,Xs)].\\
~~~~call(F,A,R1,R).        &~~~~apply(F,X,FA),         & \\
                           &~~~~apply(FA,R1,R).        & \\
%\hline
& & \\
compose(F,G,X,FGX) :-      &compose(F,G,X,FGX) :-      &fun compose(F,G,X) = F @ [G @ [X]]. \\
~~~~call(G,X,GX),          &~~~~apply(G,X,GX),         & \\
~~~~call(F,GX,FGX).        &~~~~apply(F,GX,FGX).       & \\
%\hline
& & \\
converse(F,X,Y,FYX) :-     &converse(F,X,Y,FYX) :-     &fun converse(F,X,Y) = F @ [Y,X]. \\
~~~~call(F,Y,X,FYX).       &~~~~apply(F,Y,FY),         &\\
                           &~~~~apply(FY,X,FYX).       &\\
\hline
\end{tabular}
}
\caption{Comparison of \texttt{call/N}, \texttt{apply/3} and PrologPF}
\label{call_apply}
\end{table}

The above relations and functions are then tested against the following
queries in Table \ref{call_apply_queries} \cite{Nai96}:

\begin{table}[htbp]
{\footnotesize
\tt
\begin{tabular}{| l | l | l |}
\hline
   &~~~~call/N, apply/3          &~~~~~~~PrologPF \\
\hline
& & \\
1. &filter(>(5),[3,4,5,6,7],As)  &As = filter(>(5),[3,4,5,6,7])\\
2. &map(plus(1),[2,3,4],As)      &As = map(+1,[2,3,4])\\
3. &map(between(1),[2,3],As)     &$\Rightarrow$ non-deterministic function\\
4. &map(plus(1),As,[3,4,5])      &$\Rightarrow$ reversible map, plus\\
5. &map(plus(X),[2,3,4],[3,4,5]) &$\Rightarrow$ reversible plus\\
6. &map(plus(X),[2,A,4],[3,4,B]) &$\Rightarrow$ reversible plus\\
7. &map(plus(X),[A,3,4],[3,4,B]) &$\Rightarrow$ reversible plus\\
8. &foldr(append,[],[[2],[3,4],[5]],As) &As = foldr(append,[],[[2],[3,4],[5]]) \\
9. &foldr(converse(append),      &As = foldr(converse(append),\\
   &~~~~~~[],                    &~~~~~~~~~~~[],\\
   &~~~~~~[[2],[3,4],[5]],       &~~~~~~~~~~~[[2],[3,4],[5]]\\
   &~~~~~~As                     &~~~~~~~~~~).\\
   &~~~~~).                      & \\
10. &compose(map(plus(1)),       &As = map(+1) @ [foldr(append,[]) @\\
    &~~~~~~~~foldr(append,[]),   &~~~~~~~~~~~~~~~~~~[[2],[3,4],[5]]\\
    &~~~~~~~~[[2],[3,4],[5]],    &~~~~~~~~~~~~~~~].\\
    &~~~~~~~~As                  & \\
    &~~~~~~~).                   & \\
11. &foldr(compose(append,map(plus(1))), &As = foldr(compose(append, map(+1)),\\
    &~~~~~~[],                           &~~~~~~~~~~~[],\\
    &~~~~~~[[2],[3,4],[5]],              &~~~~~~~~~~~[[2],[3,4],[5]]\\
    &~~~~~~As                            &~~~~~~~~~~). \\
    &~~~~~).                             & \\
12. &map(plus,[2,3,4],As).       & As = map(+,[2,3,4]).\\
\hline
\end{tabular}
}
\label{call_apply_queries}
\caption{Queries from \cite{Nai96} for \texttt{call/N}, \texttt{apply/3}, PrologPF}
\end{table}

With the syntax shown in the right-hand column, PrologPF can support the 
functional examples given in \cite{Nai96} with the exception of those requiring
multiple answers (3) or reversible functions (4-7).  \texttt{Call/N} does not
provide reversible functions (4-7) or permit general higher-order programming
as in (11-12).  \texttt{Apply/3} does not provide reversible functions (4-7).
A discussion of the significant features of each example is given below
(and in [Nai96]), followed here
by some more examples highlighting the capabilities of PrologPF.

\begin{enumerate}
\item{\texttt{filter($>$(5),[3,4,5,6,7],As)}\\
  The function $>$ passed to \texttt{filter} is curried, representing the 
  boolean function $\lambda x \rightarrow (5 > x)$.  The higher-order function
  \texttt{filter} applies this argument to \texttt{[3,4,5,6,7]}, returning
  \texttt{[3,4]}.  The example exercises the definition of higher-order functions
  and currying.}
\item{\texttt{map(plus(1),[2,3,4],As)}\\
  In a similar fashion to example 1, the curried function \texttt{plus(1)} is
  passed to the higher-order function \texttt{map}.  In PrologPF the function and
  predicate namespaces are different (see section \ref{bool}), so the plus function
  can be given the name \texttt{+} rather than a special relation being needed.  The
  PrologPF library includes the definitions of all the arithmetic functions, e.g.
  \mbox{\texttt{fun +(X,Y) = if (Z is X+Y) then Z else fail.}}.  The \texttt{is}
  relation is redundant in PrologPF.}
\item{\texttt{map(between(1),[2,3],As)}\\
  The relation \texttt{between(I,J,X)} has multiple solutions, and its call
  from within a functional expression in PrologPF (such as 
  \mbox{\texttt{if between(1,9,N) then N else 0}}) would ensure deterministic execution of
  the predicate.  This would enforce a single solution or failure.  Example 3 has no
  equivalent in the functional component of PrologPF, as that would conflict with
  the implementation on the Delphi Machine.}
\item{\texttt{map(plus(1),As,[3,4,5])}\\
  Examples 4 through 7 require the functions \texttt{map} or \texttt{plus}
  to be reversible.  None of \texttt{call/N}, \texttt{apply/3} or PrologPF
  provides support for reversible functions.}
\item{\texttt{map(plus(X),[2,3,4],[3,4,5])}\\
  See 4 above.}
\item{\texttt{map(plus(X),[2,A,4],[3,4,B])}\\
  See 4 above.}
\item{\texttt{map(plus(X),[A,3,4],[3,4,B])}\\
  See 4 above.}
\item{\texttt{foldr(append,[],[[2],[3,4],[5]],As)}\\
  The higher-order function \texttt{foldr} accepts a function abstraction
  (in this case the function \texttt{append}) and recursively applies it to
  the argument list, treating the argument \texttt{[]} and the final element.
  With \texttt{call/N} and \texttt{apply/3},
  the first call to \texttt{append} is with the last element of the list of lists 
  and \texttt{[]}, e.g. \texttt{append([5],[],R)}, where \texttt{R} is an intermediate
  result. Similarly, PrologPF stacks the intermediate result of
  \texttt{append([5],[])}.  Each call to \texttt{append} is with both required
  arguments ground, and \texttt{call/N}, \texttt{apply/3} and PrologPF provide the
  flattened solution \mbox{\texttt{[2,3,4,5]}}.}
\item{\texttt{foldr(converse(append),[],[[2],[3,4],[5]],As)}\\
  The example proceeds in a similar manner to example 8, with the function
  abstraction provided by \texttt{converse(append)}.  When called by
  \texttt{foldr}, the abstraction is passed both required arguments which are
  appended in reverse, resulting in the solution \texttt{[5,3,4,2]}.}
\item{\texttt{compose(map(plus(1)),foldr(append,[]),[[2],[3,4],[5]],As)}\\
  This is a more complex combination of currying and higher-order
  functions, but with similar system requirements to examples 8 and 9.
  \texttt{map(plus(1))} increments each member of a list, while 
  \texttt{foldr(append,[])} flattens a list of lists, so the term represents
  \mbox{\textit{increment\_{}list(flatten\_{}list([[2],[3,4],[5]]))}}.  This can be represented
  more naturally in PrologPF than in the flat syntax with \texttt{call/N} and
  \texttt{apply/3}.}
\item{\texttt{foldr(compose(append,map(plus(1))),[],[[2],[3,4],[5]],As)}\\
  This example is evaluated successfully with \texttt{apply/3} and in PrologPF,
  but \textbf{not} with \texttt{call/N}.  The composition of \texttt{append} and
  \texttt{map(plus(1))} results in a function which increments the elements of an
  argument list, and returns a function which prepends that result onto its
  argument (i.e. \texttt{compose(append,map(+1)) @ [[1,2,3]]} \\
  \mbox{$\leadsto \lambda x \rightarrow$ \texttt{append([1,2,3],$x$)}}).  This abstraction
  can be passed to \texttt{foldr} to be recursively applied to the argument list
  \texttt{[[2],[3,4],[5]]} and \texttt{[]} producing \texttt{[3,4,5,6]}.
  The problem that \texttt{call/N} has with this example stems from the fact that
  an intermediate result is produced which is a function abstraction.
  \texttt{Call/N} requires that the right number of arguments must be given for the
  call to work correctly.  For example, \texttt{call(plus(1),2,Z)} works correctly
  giving \texttt{Z = 3}, but \texttt{call(plus,1,X)} results in an error or fails.
  This limitation of \texttt{call/N} provides the motivation for Naish \cite{Nai96}
  to recommend \texttt{apply/3} in which every application is to one argument and
  a closure is returned if the function is defined with more.}
\item{\texttt{map(plus,[2,3,4],As)}\\
  In this case, the application of \texttt{map} must return an array of
  function abstractions, highlighting the weakness of \texttt{call/N} as in example
  11.  PrologPF and \texttt{apply/3} both produce the expected result, which can be tested 
  in a query such as\\
  \mbox{\texttt{?- map(plus,[2,3,4],[Fa,Fb,Fc]), apply(Fb,5,Z).}}\\ or for PrologPF\\
  \mbox{\texttt{?- [Fa,Fb,Fc] = map(plus,[2,3,4]), Z = Fb @ [5].}}\\ giving the
  solution \mbox{\texttt{Z = 8}}.}
\end{enumerate}

The examples above illustrate the limitations of \texttt{call/N} and show the similarities
of \texttt{apply/3} and PrologPF for non-deterministic functions.  Other examples will highlight
the syntactic advantages of PrologPF over \texttt{apply/3}, in
table \ref{prologpf_syntax_advantages}.

\begin{table}[htbp]
{\footnotesize
\tt
\begin{tabular}{| l | l | l |}
\hline
   &~~~~~~~apply/3                &~~~~~~~~PrologPF \\
\hline
& & \\
1. &inc(X,Y) :- Y is X+1.            &fun inc(X) = X+1.\\
2. &fact(1,1).                       &fun fact(1) = 1;\\
   &fact(X,Y) :- X $\backslash$== 1, &~~~~fact(N) = N * fact(N-1).\\
   &~~~~~~~~~~~~~X1 is X-1,       &\\
   &~~~~~~~~~~~~~fact(X1,Y1),     &\\
   &~~~~~~~~~~~~~Y is X * Y1.     &\\
3. &apply4(F,A1,A2,R) :-          &F = plus, Z = F @ [1,2].\\
   &~~~~apply(F,A1,F1),           &\\
   &~~~~apply(F1,A2,R).           &\\
   &F = plus, apply4(plus,1,2,Z). &\\
4. &divby2(X,Y) :- Y is X / 2.    &map(lambda([X],X/2),[2,4,6]).\\
   &map(div\_{}by\_{}2,[2,4,6]).  &\\
5. &                              &fun div\_{}by\_{}n(N) = lambda([X],X/N).\\
   &                              &Z = div\_{}by\_{}n(2) @ [10].\\
6. &fib(0,0).                     &fun fib(0) = 0;\\
   &fib(1,1).                     &~~~~fib(1) = 1;\\
   &fib(N,M) :- N > 1,            &~~~~fib(N) = fib(N-2) + fib(N-1).\\
   &~~~~~~~~~~~~N2 is N-2,        &\\
   &~~~~~~~~~~~~fib(N2,M2),       &\\
   &~~~~~~~~~~~~N1 is N-1,        &\\
   &~~~~~~~~~~~~fib(N1,M1),       &\\
   &~~~~~~~~~~~~M is M2 + M1.     &\\
7. &ffib(F,0,M) :- apply(F,0,M).  &fun ffib(F,0) = F @ [0].\\
   &ffib(F,1,M) :- apply(F,1,M).  &~~~~ffib(F,1) = F @ [1].\\
   &ffib(F,N,M) :- N > 1,         &~~~~ffib(F,N) = F @ [ffib(F,N-2) + ffib(F,N-1)].\\
   &~~~~~~~~~~~~~~~N2 is N-2,     &\\
   &~~~~~~~~~~~~~~~ffib(F,N2,M2), &\\
   &~~~~~~~~~~~~~~~N1 is N-1,     &\\
   &~~~~~~~~~~~~~~~ffib(F,N1,M1), &\\
   &~~~~~~~~~~~~~~~MM is M2 + M1, &\\
   &~~~~~~~~~~~~~~~apply(F,MM,M). &\\
\hline
\end{tabular}
}
\label{prologpf_syntax_advantages}
\caption{Further programming examples showing PrologPF capabilities}
\end{table}

\begin{enumerate}
\item{\texttt{Apply/3} consistently treats all functions as relations, such
  that the flat form of arithmetic expressions is retained with the \texttt{is}
  relation, as in the example with the definition of \texttt{inc}.  The functional
  support in PrologPF allows direct use of nested arithmetic expressions, so the
  \texttt{is} relation is redundant.  In fact, if \texttt{is} appears in a
  PrologPF goal with an arithmetic argument, the argument will be evaluated before
  unification with the corresponding \texttt{is} formal parameter.  This means that
  \texttt{Z is 1 + 2} $\equiv$ \texttt{R = 1 + 2, Z is R}.  \texttt{is}
  has quite asymmetric functionality
  in which the first argument must be a number or a variable while
  the second argument can also be an arithmetic expression.
  \texttt{(1 + 2) is Z} is not permitted in standard Prolog both for
  \texttt{Z} a variable or with \texttt{Z} instantiated to a number.  In PrologPF the
  use of \texttt{=} with library functions provides more consistent support for
  arithmetic,  allowing both \texttt{Z = 1 + 2} and \texttt{(1 + 2) = Z}.  The
  bracketed terms are for clarity, and \texttt{1 + 2 = Z} is equally acceptable.}
\item{The example of the factorial function \texttt{fact} shows that deterministic
  functions in the relational style must have guard conditions in subsequent clauses
  (i.e. \texttt{X $\backslash$\== 1}) to prevent erroneous non-deterministic execution.
  For more complex functions the conditions can obscure the meaning of the code, and
  Prolog's \textit{cut} is used to provide an efficient solution.  \texttt{apply/3}
  does not attempt to address the presence on \texttt{cut} to ensure determinism
  in functions, while PrologPF has consistent deterministic functional evaluation.}
\item{The use of \texttt{apply/3} provides consistent support for higher-order
  functional programming, but suffers from the implicit treatment of all function
  applications as nested applications to one argument and the flat representation of
  application terms.  The example shows the application of an arity/2 function to
  two arguments, and Naish \cite{Nai96} suggests the definition of an auxilliary
  relation \texttt{apply4} to mitigate this difficulty.  PrologPF allows the
  application of functions to an arbitrary number of arguments in a single term.}
\item{Without nameless functions, the use of \texttt{apply/3} requires that defined
  functions are created for each requirement, and the chosen name used in the place
  of the lambda expression in PrologPF.  The example shows the specification of
  a function which divides-by-two.  The issue with \texttt{apply/3} is mitigated by
  the use of currying, such that if the required function were times-by-two, then
  a curried application (e.g. \texttt{times(2)}) could by used.  In general, however,
  an auxilliary fact will be needed, as the example shows.}
\item{The use of defined functions as an alternative to lambda-expressions with
  \texttt{apply/3} is unsatisfactory where the lambda-expression contains free
  logical
  variables.  The example shows such an expression in the definition of
  \texttt{div\_{}by\_{}n}, and the issue would similarly arise within a goal
  such as \texttt{?- N = @, Z = lambda([X],X/Z) @ [10].}  The implementation with
  \texttt{apply/3} would require the use of the Prolog extra-logical relation
  \texttt{assert} or the accumulation of free variables as additional arguments to
  the auxilliary functions.}
\item{The eager argument evaluation semantics of PrologPF is equivalent to
  the flattened form of Prolog relations used with \texttt{apply/3}.  The
  example of the fibonacci function show the syntax of PrologPF to be a better
  match to the requirement.}
\item{The awkwardness of the flattened form with \texttt{apply/3} is exacerbated
  when nested expressions and higher-order applications appear in the
  function definition.  The example gives a modified fibonacci function where an
  additional parameter specifies a function (\texttt{F}) to be applied to the
  sub-terms before summation in the recursive case.  The \texttt{apply/3} example
  of \texttt{ffib} illustrates the following differences with PrologPF:
  \begin{enumerate}
  \item{The condition \texttt{N>1} is required as \texttt{apply/3} has no
    special consideration for deterministic execution, and assumes use of 
    additional conditions or \textit{cut}.}
  \item{All expressions with \texttt{apply/3} retain their flat Prolog form,
    leading to an unwieldy syntax for expressions which would naturally be
    nested.}
  \item{Arithmetic with \texttt{apply/3} relies upon the use of the special
    Prolog relation \texttt{is}.  In PrologPF arithmetic expressions can appear
    anywhere as a valid argument term, and will be reduced before the term
    is unified with the corresponding formal argument.}
  \item{Function application in PrologPF can be either explicit with the
    \texttt{@} operator, or implicit by using a defined function name in
    a compound argument term.  The latter case is defined to be syntactic
    sugaring for the former.  The definition of \texttt{ffib} using 
    \texttt{apply/3} differentiates between the application of a higher-order
    term represented by the variable \texttt{F} in \texttt{apply(F,MM,M)} and
    the recursive call to the function \texttt{ffib} in \texttt{ffib(F,N2,M2)}.
    For consistent use of \texttt{apply/3}, the recursive call would be
    replaced by \texttt{apply(ffib(F,N2),M2)} which would be converted by
    \texttt{apply/3} to the call \texttt{ffib(F,N2,M2)}.  It is unclear whether
    it is better to make consistent use of \texttt{apply/3} in higher-order
    functions and render the non-curried calls more obscure, or whether a mix
    of \texttt{apply/3} and normal relation calls should be used.}
  \end{enumerate}}
\end{enumerate}

%%%%%%%%%%%%%%%%%%%%%%%
\section{Conclusions} %
%%%%%%%%%%%%%%%%%%%%%%%

Higher-order functions can be neatly integrated with Prolog's relations
with a deterministic evaluation semantics compatible with the requirements
of a Delphi implementation.

Examples given in this and the following two chapters show the capabilities
chosen for implementation in PrologPF to be sufficient to express a wide range
of programs without resorting to artificial or obscure coding devices.

The capabilities of PrologPF, including the definition and application of
functions, the call-once semantics of the \texttt{if} condition, and the
use of boolean functions as relations, have proved sufficient to preclude the
need for \textit{cut} in all the test programs reviewed to date.

%%%%%%%%%%%%%%%%%%%
\section{Summary} %
%%%%%%%%%%%%%%%%%%%

The functional component of PrologPF extends
the Prolog language in the following ways:
\begin{itemize}
\item{The definition of functions through the \texttt{fun} relation}
\item{The application of functions through the \texttt{@} operator}
\item{Support for higher-order functional programming through the
  use of lambda-expressions and currying}
\item{A general strict functional evaluation semantics with
  the single exception of
  a pre-defined \texttt{if} function}
\item{Use of relations within functions is limited to the condition
  argument of the \texttt{if} function, where deterministic search for
  the first solution is enforced}
\item{Support for boolean functions to be treated as relations}
\end{itemize}

These features have proved consistent in use and sufficient to implement
a wide range of sample programs without resorting to \textit{cut}.

The defined semantics permit
an efficient implementation on an extended Delphi Machine, 
where function applications
embedded within a Prolog program are compiled to direct machine-code
calls.  Such an implementation has been produced in PrologPF.
