\chapter{Conclusions}
\label{conclusions}

Programs compiled with the PrologPF compiler can exploit the processing
power available in a LAN or WAN of general purpose workstations, without
requiring programmer annotations to the code to guide the parallelisation.

The breadth-first partitioning strategy minimises the communication
necessary for distributed execution, such that the technique is
optimised for systems with many processors but significant communications
delay, such as the Internet.
The combined use of oracles and the partitioning primitive \texttt{kappa}
introduces communication during the execution to effect a dynamic redistribution
of the workload.
The use of oracles provides an efficient
means of recovery from path processor failure, supporting the use of
the technique in a widely distributed system.  The single systems
image provided by the combination of the PrologPF compiler and the
skynet control system transform a generally idle network of
workstations into a usable super-computer.

PrologPF provides effective speedup for large\footnote{In this context, \textit{large}
implies a single-cpu runtime greater than a few seconds.}
problems containing sufficient OR-parallelism.  For smaller problems, the runtime
taken to generate sufficient oracles for distribution to the available path processors
will dominate the overall runtime.
Given a large problem, the speedup achieved by PrologPF is determined by the
evenness of the balance of the workload assigned to the path processors.

The implementation of the oracle support in PrologPF imposes a 9\% to 16\%
overhead on the execution of the compiled program on a single cpu.  This overhead
may be acceptable for the use of the compiled binaries in both a single-cpu and
distributed environment, or the support for oracles can be more efficiently implemented
in the virtual machine rather than the source transformation technique used with
\texttt{prologpf}.

The initial phase of the breadth-first partitioning strategy produces
a number of open oracles at the selected depth.  The distribution of
work beneath these oracles can be quantified with the simulation of 
a subsequent execution of the problem with the number of path processors
$G$ set to the number of open oracles $S$ discovered at the depth limit
$L$ in an earlier execution with $G=1$.
An analysis of some commonly available Prolog benchmark programs shows
that many of those oracles will refer to small subtrees, with
a sparse distribution of oracles referring to large subtrees
representing most of the available work.
To achieve effective partitioning of the search tree, the depth limit
at which open oracles are found must generate a sufficient number of
oracles referring to large subtrees, such that all path processors can
expect to receive a number of these sizeable oracles.  The general
issue for oracle-based scheduling strategies such as the 
breadth-first partitioning is that of the
\textit{poisoned oracle}.  For any strategy without subsequent work
splitting, a poisoned oracle is likely to be an oracle which refers
to a very large subtree, such that the uninterrupted sequential search of
that subtree dominates the overall parallel runtime.  For other
strategies used with oracles requiring communication, 
such as the automatic partitioning strategy of DelphiKS, a poisoned oracle
is that referring to a very small or nil subtree.

An effective work estimation function could lead to better balancing of
the work assignment to the path processors.  The generally sparse
and random distribution of the large oracles means that work estimation
based upon the arithmetic mean of the oracle neighbours will not
succeed.  Partially searching the subtree but limiting the number of
choice-points traversed similarly relies upon a limited measure of
workload beneath certain oracles being used as an estimate for others.

The breadth-first partitioning strategy tested with PrologPF succeeds because
a large number of oracles can be efficiently generated and allocated, and the
process of following an assigned oracle within a path processor is
very efficient.  The set of oracles assigned to a given path processor will
typically include many referring to very small subtrees, but the within the
set a few will refer to larger subtrees, providing sufficient work for effective
partitioning.

The meta-logical predicate \textit{cut}\footnote{Written ``\texttt{!}'' in Prolog.}
 used in sequential Prolog programs to
prune the search tree does not have an efficient implementation in a
distributed OR-parallel environment.  The propagation of a \textit{cut} in the search
tree of one path processor requires the communication to the other path
processors searching affected subtrees.  A program containing a small procedure
with \textit{cut} may encounter that predicate thousands of times each second.
PrologPF provides a useful programming system for logic programs that in Prolog
would contain \textit{cut} by supporting deterministic execution through the
use of higher-order functional programming, explicit negation through the use
of boolean functions, and permitting \textit{cut} in deterministic relational
procedures.

The implementation of higher-order functions in PrologPF has been achieved while
maintaining upwards compatibility with standard Prolog.  A consistent syntax is
possible for the combined styles, permitting reduction of functional terms as
arguments to relations and requiring relational goals as conditions in functions.
Allowing the definition of functions and relations of the same name but different arities
within the same PrologPF program facilitates the straightforward mapping of existing
deterministic library relations into new functions.  All function calls in PrologPF
appear in arguments to relations. Function failure can be supported as an exception
propagated to the outermost call in the argument term and
treated as unification failure leading to the failure of the underlying relational
subgoal.

PrologPF produces efficient compiled code, with the compiler built upon the
\texttt{wamcc} Prolog to C compiler \cite{CD95}.  Efficient support for
oracles can be provided though the use of simple primitives, \texttt{o\_{}kbuild}
and \texttt{o\_{}kfollow}, automatically embedded in the user program
by a pre-processing pass of the compiler.  The definition of the
primitives permits their inline implementation as C macros, and the source is
given in Appendix \ref{kappa_macro}.

For the simple one-time breadth-first partitioning scheduling algorithm tested with
PrologPF, an equivalent behaviour can be achieved through the use of a novel primitive
\texttt{kappa}.  The primitive, embedded as a proposition in the user program, performs
dynamic pruning resulting in the same partitioning of the search as PrologPF but without
the intermediate use of oracles.  The absence of oracles means that basic refinements
to the technique available to a future development of PrologPF may not be feasible with
\texttt{kappa}.  Implemented as a two-line C macro, \texttt{kappa} can be used implement
the BFP strategy with
most standard Prolog compilers.

The use of oracles can be combined with the partitioning primitive
\texttt{kappa} to support scheduling strategies using the recursive reassignment of
work from busy to idle path processors.  
While most OR-parallel Prolog systems assume a work assignment rate of many thousands
of times per second, the partitioning techniques discussed in this dissertation are
effective with work reassignment occurring orders of magnitude less frequently.

The use of oracles as the fundamental mechanism to communicate work for distributed
execution provides an efficient means of recovery from processor failure.  The work
can be reassigned to an alternative processor without the other path processors in the
group being affected and with a minimal impact on the total execution time.

%%%%%%%%%%%%%%%%%%%%%%%
\section{Future work} %
%%%%%%%%%%%%%%%%%%%%%%%

The scheduling strategy used in PrologPF assumes the work beneath the
discovered open oracles is randomly distributed, and no attempt is made to
associate a work estimate with each oracle.  If an effective estimation technique
could be found, then the effectiveness of the one-time allocation of work would be
greatly improved.

In PrologPF, the parallelisation of the user program is limited to the 
logical component.  Functional reduction within a specific subgoal is executed
sequentially.  Further research may lead to an effective method for extending the
use of oracles to parallelise functional execution.

PrologPF supports the use of \textit{failure} as an uncatchable exception within
a functional reduction, which is propagated to the top level function call and
leads to failure of the underlying subgoal.  More flexible support might be
provided through the general provision of exception support in the functions,
unified with the exception support of standard Prolog.






