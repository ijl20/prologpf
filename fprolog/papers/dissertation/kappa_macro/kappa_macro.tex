\chapter[Source code for \texttt{kappa}]{Source code for the parallelisation primitive}
\label{kappa_macro}

This appendix gives the source code for the simple parallelisation primitive
supporting oracles and kappa.

To embed the parallelisation support, each user clause is modified to include
a special goal \texttt{o\_{}kbuild(N)} where \texttt{N} is the index of the clause in
the current procedure.  This example program is transformed as
follows:
\begin{verbatim}
a(X) :- b(X), c(X).
a(X) :- d(X).
a(z).

b(X) :- c(X).
b(b).

c(z).
\end{verbatim}
This program becomes:
\begin{verbatim}
a(X) :- o_kbuild(1),b(X), c(X).
a(X) :- o_kbuild(2),d(X).
a(z) :- o_kbuild(3).

b(X) :- o_kbuild(1),c(X).
b(b) :- o_kbuild(2).

c(z) :- o_kbuild(1).
\end{verbatim}

The primitive \texttt{o\_{}kbuild} is implemented as a 'C' macro and treated as
a goal by a stub Prolog procedure:
\begin{verbatim}
o_kbuild(N) :- pragma_c(o_kbuild1).
o_kbuild(_) :- pragma_c(o_kbuild2).
\end{verbatim}

The 'C' macros defining \texttt{o\_{}kbuild1} and \texttt{o\_{}kbuild2} are
as follows:
\begin{verbatim}
#define MAXORC  10000     /* maximum length of oracle            */

static int defer = 0;     /* flag to defer interrupt handling    */
static int skip = 0;      /* flag to skip prev port on interrupt */

static int orc[MAXORC];   /* array to hold orc as it is built    */

static int b_depth;       /* current BUILD or-depth              */

static int orc_l0;        /* depth limit L0                      */
static int orc_l;         /* depth limit L                       */
static int orc_g;         /* group count G                       */
static int orc_n;         /* unique processor number N           */
static int orc_s;         /* count of ports S                    */
static int orc_length;    /* length of oracle to follow          */

#define o_kbuild1      
    { int index;       
      ++b_depth;       
      Deref(A(0),word,tag,adr) 
      index = UnTag_INT(word);  /* index := argument to o_kbuild */
      if (b_depth <= orc_l0) { if (index != orc[b_depth]) fail;}
      else { if (b_depth <= orc_length) 
               { if (index < orc[b_depth]) fail;
                 if (index > orc[b_depth]) orc_length = 0;
               }                               
             orc[b_depth] = index;             
             if (b_depth == orc_l) /* if at partitioning depth...*/
               { if (skip) { skip = 0; fail; } 
                 orc_s++;                      
                 if (orc_s % orc_g != orc_n) fail;
                 if (defer) { defer = 0; send_oracle(); fail; }  
               }                      
           }                          
    }

#define o_kbuild2     /* called on backtracking through o_kbuild */
    {                                 
      b_depth--;                      
      fail;                           
    }
\end{verbatim}

Similar parallelisation primitives which exploit more complex transformations of the
user program are possible (for example see Chapter \ref{bfp_depth}).  Also the
primitive described above could equally be implemented wholly as a 'C' macro without
the use of the stub Prolog procedure.  The most efficient implementation might use
a modified abstract machine. However, the implementation described above
is portable to any Prolog compiler supporting embedded 'C' code, and the simple
definition seems to produce an acceptable overhead of approximately 10\%.
